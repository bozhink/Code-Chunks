\documentclass[a4paper,10pt]{article}

\usepackage{ucs}
\usepackage[utf8x]{inputenc}
\usepackage{amsmath}
\usepackage{amsfonts}
\usepackage{amssymb}
\usepackage[bulgarian,english]{babel}
\usepackage[T1,T2A]{fontenc}
\usepackage[pdftex]{graphicx}

\author{Божин Ангелов Караиванов}
\title{TASEP}
\date{25.4.2013}

\begin{document}
 \maketitle
Разглеждаме случая $L=4$, $N=2$. Използваме следните означения на състоянията $C_{i}$
\begin{equation}
\begin{split}
C_{1} &= (1, 1, 0, 0),\\
C_{2} &= (0, 1, 1, 0),\\
C_{3} &= (0, 0, 1, 1),\\
C_{4} &= (1, 0, 0, 1),\\
C_{5} &= (1, 0, 1, 0),\\
C_{6} &= (0, 1, 0, 1).\\
\end{split}
\end{equation}

Връзката между началните $C_i$ и крайните $C_j$ състояния записваме във вида
\begin{equation}
C_i \rightarrow T^{j}_{\phantom{j}i} C_j,
\end{equation}
където $$T^{j}_{\phantom{j}i} = \frac{1}{4}\left({T_1}^{j}_{\phantom{j}i} + {T_2}^{j}_{\phantom{j}i} + {T_3}^{j}_{\phantom{j}i} + {T_4}^{j}_{\phantom{j}i}\right)$$
са компонентите на матрицата на прехода, на която сумата от елементите по редове трябва да е единица,
а отделните матрици $T_{k},\ k=1,2,3,4$ са матриците на преходите при фиксирана начална връзка за
стартиране на алгоритъма на обновяване, като $k=1$ съответства на начална връзка $(1,2)$,
$k=2$ -- на начална връзка $(2,3)$ и т. н.

Вероятностите $P_i$ за поява на дадено състояние $C_i$ намираме по формулата
\begin{equation}
 P_{i'} = {T^{\mathtt{t}}}^{i}_{\phantom{i}i'} P_{i},
\end{equation}
откъдето, като наложим условието за стационарност $P_{i'}=P_{i}$, ще получим
стойностите на вероятностите $P_{i}$ в еднопараметрична форма.
По подобен начин можем да определим тези вероятности за всеки един
от четирите различин случая на начална връзка.


\paragraph{Начална връзка $(4,1)$}
\begin{equation}
\begin{split}
C_1 \xrightarrow{(4,1)}& C_1\\
    \xrightarrow{(3,4)}& C_1\\
    \xrightarrow{(2,3)}& (1-p)\,C_1+p\,C_5\\
    \xrightarrow{(1,2)}& (1-p)\,C_1+p(1-\tilde{p})\,C_5+p\tilde{p}\,C_2.
\end{split}
\end{equation}
%
\begin{equation}
\begin{split}
C_2 \xrightarrow{(4,1)}& C_2\\
    \xrightarrow{(3,4)}& (1-p)\,C_2 + p\,C_6\\
    \xrightarrow{(2,3)}& (1-p)\,C_2 + p(1-\tilde{p})\,C_6 + p\tilde{p}\,C_3\\
    \xrightarrow{(1,2)}& (1-p)\,C_2 + p(1-\tilde{p})\,C_6 + p\tilde{p}\,C_3.
\end{split}
\end{equation}
%
\begin{equation}
\begin{split}
C_3 \xrightarrow{(4,1)}& (1-p)\,C_3 + p\,C_5\\
    \xrightarrow{(3,4)}& (1-p)\,C_3 + p(1-\tilde{p})\,C_5 + p\tilde{p}\,C_4\\
    \xrightarrow{(2,3)}& (1-p)\,C_3 + p(1-\tilde{p})\,C_5 + p\tilde{p}\,C_4\\
    \xrightarrow{(1,2)}& (1-p)\,C_3 + p(1-p)(1-\tilde{p})\,C_5 + \\
    & + p(1-p)\tilde{p}\,C_4 + p^2(1-\tilde{p})\,C_2 + p^2\tilde{p}\,C_6.
\end{split}
\end{equation}
%
\begin{equation}
\begin{split}
C_4 \xrightarrow{(4,1)}& C_4\\
    \xrightarrow{(3,4)}& C_4\\
    \xrightarrow{(2,3)}& C_4\\
    \xrightarrow{(1,2)}& (1-p)\,C_4 + p\,C_6.
\end{split}
\end{equation}
%
\begin{equation}
\begin{split}
C_5 \xrightarrow{(4,1)}& C_5\\
    \xrightarrow{(3,4)}& (1-p)\,C_5 + p\,C_4\\
    \xrightarrow{(2,3)}& (1-p)\,C_5 + p\,C_4\\
    \xrightarrow{(1,2)}& (1-p)^2\,C_5 + p(1-p)\,(C_2 + C_4) + p^2\, C_6.
\end{split}
\end{equation}
%
\begin{equation}
\begin{split}
C_6 \xrightarrow{(4,1)}& (1-p)\,C_6 + p\,C_1\\
    \xrightarrow{(3,4)}& (1-p)\,C_6 + p\,C_1\\
    \xrightarrow{(2,3)}& (1-p)^2\,C_6 + p(1-p)\,(C_3 + C_1) + p^2\,C_5\\
    \xrightarrow{(1,2)}& (1-p)^2\,C_6 + p(1-p)\,(C_3 + C_1) + p^2(1-\tilde{p}) C_5 + p^2\tilde{p}\,C_2.
\end{split}
\end{equation}

Матрицата на преход има вида
{\small
\begin{equation}
\begin{split}
&T_{4} = \\
&\begin{pmatrix}
 1-p & p\,\tilde{p} & 0 & 0 & p\,\left( 1-\tilde{p}\right)  & 0\cr
 0 & 1-p & p\,\tilde{p} & 0 & 0 & p\,\left( 1-\tilde{p}\right) \cr
 0 & {p}^{2}\,\left( 1-\tilde{p}\right)  & 1-p & \left( 1-p\right) \,p\,\tilde{p} & \left( 1-p\right) \,p\,\left( 1-\tilde{p}\right)  & {p}^{2}\,\tilde{p}\cr
 0 & 0 & 0 & 1-p & 0 & p\cr
 0 & \left( 1-p\right) \,p & 0 & \left( 1-p\right) \,p & {\left( 1-p\right) }^{2} & {p}^{2}\cr
 \left( 1-p\right) \,p & {p}^{2}\,\tilde{p} & \left( 1-p\right) \,p & 0 & {p}^{2}\,\left( 1-\tilde{p}\right)  & {\left( 1-p\right) }^{2}
\end{pmatrix}.
\end{split}
\end{equation}
}

Вероятностите $P_i$ за поява на състоянието $C_i$ можем за изразим в удобен вид чрез $P_6$
както следва
\begin{equation}
 P_1 = (1-p)\,P_6,
\end{equation}
\begin{equation}
 P_2 = \frac{2(1-p) + p\tilde{p}}{2-p-\tilde{p}(1-\tilde{p})} P_6,
\end{equation}
\begin{equation}
 P_3 = \frac{(1-p)(2-p)+\tilde{p}(1-p+\tilde{p})}{2-p-\tilde{p}(1-\tilde{p})} P_6,
\end{equation}
\begin{equation}
 P_4 = \frac{(1-p)[2-2p+p^2+2\tilde{p}(\tilde{p}-p)]}{2-p-\tilde{p}(1-\tilde{p})} P_6,
\end{equation}
\begin{equation}
 P_5 = \frac{(1-\tilde{p})[2-2p+p^2+\tilde{p}(\tilde{p}-p)]}{2-p-\tilde{p}(1-\tilde{p})} P_6.
\end{equation}
%
%
%
%
\paragraph{Начална връзка $(3,4)$}
\begin{equation}
\begin{split}
C_1 \xrightarrow{(3,4)}& C_1\\
    \xrightarrow{(2,3)}& (1-p)\,C_1 + p\,C_5\\
    \xrightarrow{(1,2)}& (1-p)\,C_1 + p(1-\tilde{p})\,C_5 + p\tilde{p}\,C_2\\
    \xrightarrow{(4,1)}& (1-p)\,C_1 + p(1-\tilde{p})\,C_5 + p\tilde{p}\,C_2.
\end{split}
\end{equation}
%
\begin{equation}
\begin{split}
C_2 \xrightarrow{(3,4)}& (1-p)\,C_2 + p\,C_6\\
    \xrightarrow{(2,3)}& (1-p)\,C_2 + p(1-\tilde{p})\,C_6 + p\tilde{p}\,C_3\\
    \xrightarrow{(1,2)}& (1-p)\,C_2 + p(1-\tilde{p})\,C_6 + p\tilde{p}\,C_3\\
    \xrightarrow{(4,1)}& (1-p)\,C_2 + p(1-p)(1-\tilde{p})\,C_6 + \\
    & + p(1-p)\tilde{p}\,C_3 + p^2(1-\tilde{p})\,C_1 + p^2\tilde{p}\,C_5.
\end{split}
\end{equation}
%
\begin{equation}
\begin{split}
C_3 \xrightarrow{(3,4)}& C_3\\
    \xrightarrow{(2,3)}& C_3\\
    \xrightarrow{(1,2)}& C_3\\
    \xrightarrow{(4,1)}& (1-p)\,C_3 + p\,C_5.
\end{split}
\end{equation}
%
\begin{equation}
\begin{split}
C_4 \xrightarrow{(3,4)}& C_4\\
    \xrightarrow{(2,3)}& C_4\\
    \xrightarrow{(1,2)}& (1-p)\,C_4 + p\,C_6\\
    \xrightarrow{(4,1)}& (1-p)\,C_4 + p(1-\tilde{p})\,C_6 + p\tilde{p}\,C_1.
\end{split}
\end{equation}
%
\begin{equation}
\begin{split}
C_5 \xrightarrow{(3,4)}& (1-p)\,C_5 + p\,C_4\\
    \xrightarrow{(2,3)}& (1-p)\,C_5 + p\,C_4\\
    \xrightarrow{(1,2)}& (1-p)^2\,C_5 + p(1-p)\,(C_2 + C_4) + p^2\, C_6\\
    \xrightarrow{(4,1)}& (1-p)^2\,C_5 + p(1-p)\,(C_2 + C_4) + p^2(1-\tilde{p})\,C_6 + p^2\tilde{p}\,C_1.
\end{split}
\end{equation}
%
\begin{equation}
\begin{split}
C_6 \xrightarrow{(3,4)}& C_6\\
    \xrightarrow{(2,3)}& (1-p)\,C_6 + p\,C_3\\
    \xrightarrow{(1,2)}& (1-p)\,C_6 + p\,C_3\\
    \xrightarrow{(4,1)}& (1-p)^2\,C_6 + p(1-p)\,(C_1+C_3) + p^2\,C_5.
\end{split}
\end{equation}

Матрицата на преход има вида
{\small
\begin{equation}
\begin{split}
&T_{3} = \\
&\begin{pmatrix}
 1-p & p\,\tilde{p} & 0 & 0 & p\,\left( 1-\tilde{p}\right)  & 0\cr
 {p}^{2}\,\left( 1-\tilde{p}\right)  & 1-p & \left( 1-p\right) \,p\,\tilde{p} & 0 & {p}^{2}\,\tilde{p} & \left( 1-p\right) \,p\,\left( 1-\tilde{p}\right) \cr
 0 & 0 & 1-p & 0 & p & 0\cr
 p\,\tilde{p} & 0 & 0 & 1-p & 0 & p\,\left( 1-\tilde{p}\right) \cr
 {p}^{2}\,\tilde{p} & \left( 1-p\right) \,p & 0 & \left( 1-p\right) \,p & {\left( 1-p\right) }^{2} & {p}^{2}\,\left( 1-\tilde{p}\right) \cr
 \left( 1-p\right) \,p & 0 & \left( 1-p\right) \,p & 0 & {p}^{2} & {\left( 1-p\right) }^{2}
\end{pmatrix}.
\end{split}
\end{equation}
}

Вероятностите $P_i$ за поява на състоянието $C_i$ можем за изразим в удобен вид чрез $P_5$
както следва
\begin{equation}
 P_4 = (1-p)\,P_5
\end{equation}
\begin{equation}
 P_1 = \frac{2(1-p) + p\tilde{p}}{2-p-\tilde{p}(1-\tilde{p})} P_5,
\end{equation}
\begin{equation}
 P_2 = \frac{(1-p)(2-p)+\tilde{p}(1-p+\tilde{p})}{2-p-\tilde{p}(1-\tilde{p})} P_5,
\end{equation}
\begin{equation}
 P_3 = \frac{(1-p)[2-2p+p^2+2\tilde{p}(\tilde{p}-p)]}{2-p-\tilde{p}(1-\tilde{p})} P_5,
\end{equation}
\begin{equation}
 P_6 = \frac{(1-\tilde{p})[2-2p+p^2+\tilde{p}(\tilde{p}-p)]}{2-p-\tilde{p}(1-\tilde{p})} P_5.
\end{equation}
%
%
%
%
\paragraph{Начална връзка $(2,3)$}
\begin{equation}
\begin{split}
C_1 \xrightarrow{(2,3)}& (1-p)\,C_1 + p\,C_5\\
    \xrightarrow{(1,2)}& (1-p)\,C_1 + p(1-\tilde{p})\,C_5 + p\tilde{p}\,C_2\\
    \xrightarrow{(4,1)}& (1-p)\,C_1 + p(1-\tilde{p})\,C_5 + p\tilde{p}\,C_2\\
    \xrightarrow{(3,4)}& (1-p)\,C_1 + p(1-p)(1-\tilde{p})\,C_5\\
    &+ p^2(1-\tilde{p})\,C_4 + p(1-p)\tilde{p}\,C_2 + p^2\tilde{p}\,C_6.
\end{split}
\end{equation}
%
\begin{equation}
\begin{split}
C_2 \xrightarrow{(2,3)}& C_2\\
    \xrightarrow{(1,2)}& C_2\\
    \xrightarrow{(4,1)}& C_2\\
    \xrightarrow{(3,4)}& (1-p)\,C_2 + p\,C_6.
\end{split}
\end{equation}
%
\begin{equation}
\begin{split}
C_3 \xrightarrow{(2,3)}& C_3\\
    \xrightarrow{(1,2)}& C_3\\
    \xrightarrow{(4,1)}& (1-p)\,C_3 + p\,C_5\\
    \xrightarrow{(3,4)}& (1-p)\,C_3 + p(1-\tilde{p})\,C_5 + p\tilde{p}\,C_4.
\end{split}
\end{equation}
%
\begin{equation}
\begin{split}
C_4 \xrightarrow{(2,3)}& C_4\\
    \xrightarrow{(1,2)}& (1-p)\,C_4 + p\,C_6\\
    \xrightarrow{(4,1)}& (1-p)\,C_4 + p(1-\tilde{p})\,C_6 + p\tilde{p}\,C_1\\
    \xrightarrow{(3,4)}& (1-p)\,C_4 + p(1-\tilde{p})\,C_6 + p\tilde{p}\,C_1.
\end{split}
\end{equation}
%
\begin{equation}
\begin{split}
C_5 \xrightarrow{(2,3)}& C_5\\
    \xrightarrow{(1,2)}& (1-p)\,C_5 + p\,C_2\\
    \xrightarrow{(4,1)}& (1-p)\,C_5 + p\,C_2\\
    \xrightarrow{(3,4)}& (1-p)^2\,C_5 + p(1-p)\,(C_2 + C_4) + p^2\,C_6.
\end{split}
\end{equation}
%
\begin{equation}
\begin{split}
C_6 \xrightarrow{(2,3)}& (1-p)\,C_6 + p\,C_3\\
    \xrightarrow{(1,2)}& (1-p)\,C_6 + p\,C_3\\
    \xrightarrow{(4,1)}& (1-p)^2\,C_6 + p(1-p)\,C_3 + p^2\,C_5\\
    \xrightarrow{(3,4)}& (1-p)^2\,C_6 + p(1-p)\,(C_1 + C_3) + p^2(1-\tilde{p}) n \,C_5 + p^2\tilde{p}\,C_4.
\end{split}
\end{equation}

Матрицата на преход има вида
{\small
\begin{equation}
\begin{split}
&T_{2} = \\
&\begin{pmatrix}
 1-p & \left( 1-p\right) \,p\,\tilde{p} & 0 & {p}^{2}\,\left( 1-\tilde{p}\right)  & \left( 1-p\right) \,p\,\left( 1-\tilde{p}\right)  & {p}^{2}\,\tilde{p}\cr
 0 & 1-p & 0 & 0 & 0 & p\cr
 0 & 0 & 1-p & p\,\tilde{p} & p\,\left( 1-\tilde{p}\right)  & 0\cr
 p\,\tilde{p} & 0 & 0 & 1-p & 0 & p\,\left( 1-\tilde{p}\right) \cr
 0 & \left( 1-p\right) \,p & 0 & \left( 1-p\right) \,p & {\left( 1-p\right) }^{2} & {p}^{2}\cr
 \left( 1-p\right) \,p & 0 & \left( 1-p\right) \,p & {p}^{2}\,\tilde{p} & {p}^{2}\,\left( 1-\tilde{p}\right)  & {\left( 1-p\right) }^{2}
\end{pmatrix}.
\end{split}
\end{equation}
}


Вероятностите $P_i$ за поява на състоянието $C_i$ можем за изразим в удобен вид чрез $P_6$
както следва
\begin{equation}
 P_3 = (1-p)\,P_6,
\end{equation}
\begin{equation}
 P_4 = \frac{2(1-p) + p\tilde{p}}{2-p-\tilde{p}(1-\tilde{p})} P_6,
\end{equation}
\begin{equation}
 P_1 = \frac{(1-p)(2-p)+\tilde{p}(1-p+\tilde{p})}{2-p-\tilde{p}(1-\tilde{p})} P_6,
\end{equation}
\begin{equation}
 P_2 = \frac{(1-p)[2-2p+p^2+2\tilde{p}(\tilde{p}-p)]}{2-p-\tilde{p}(1-\tilde{p})} P_6,
\end{equation}
\begin{equation}
 P_5 = \frac{(1-\tilde{p})[2-2p+p^2+\tilde{p}(\tilde{p}-p)]}{2-p-\tilde{p}(1-\tilde{p})} P_6.
\end{equation}
%
%
%
%
\paragraph{Начална връзка $(1,2)$}
\begin{equation}
\begin{split}
C_1 \xrightarrow{(1,2)}& C_1\\
    \xrightarrow{(4,1)}& C_1\\
    \xrightarrow{(3,4)}& C_1\\
    \xrightarrow{(2,3)}& (1-p)\,C_1 + p\,C_5.
\end{split}
\end{equation}
%
\begin{equation}
\begin{split}
C_2 \xrightarrow{(1,2)}& C_2\\
    \xrightarrow{(4,1)}& C_2\\
    \xrightarrow{(3,4)}& (1-p)\,C_2 + p\,C_6\\
    \xrightarrow{(2,3)}& (1-p)\,C_2 + p(1-\tilde{p})\,C_6 + p\tilde{p}\,C_3.
\end{split}
\end{equation}
%
\begin{equation}
\begin{split}
C_3 \xrightarrow{(1,2)}& C_3\\
    \xrightarrow{(4,1)}& (1-p)\,C_3 + p\,C_5\\
    \xrightarrow{(3,4)}& (1-p)\,C_3 + p(1-\tilde{p})\,C_5 + p\tilde{p}\,C_4\\
    \xrightarrow{(2,3)}& (1-p)\,C_3 + p(1-\tilde{p})\,C_5 + p\tilde{p}\,C_4.
\end{split}
\end{equation}
%
\begin{equation}
\begin{split}
C_4 \xrightarrow{(1,2)}& (1-p)\,C_4 + p\,C_6\\
    \xrightarrow{(4,1)}& (1-p)\,C_4 + p(1-\tilde{p})\,C_6 + p\tilde{p}\,C_1\\
    \xrightarrow{(3,4)}& (1-p)\,C_4 + p(1-\tilde{p})\,C_6 + p\tilde{p}\,C_1\\
    \xrightarrow{(2,3)}& (1-p)\,C_4 + p(1-p)(1-\tilde{p})\,C_6\\
    & + p^2(1-\tilde{p})\,C_3 + p(1-p)\tilde{p}\,C_1 + p^2\tilde{p}\,C_5.
\end{split}
\end{equation}
%
\begin{equation}
\begin{split}
C_5 \xrightarrow{(1,2)}& (1-p)\,C_5 + p\,C_2\\
    \xrightarrow{(4,1)}& (1-p)\,C_5 + p\,C_2\\
    \xrightarrow{(3,4)}& (1-p)^2\,C_5 + p(1-p)\,C_4 + p(1-p)\,C_2 + p^2\,C_6\\
    \xrightarrow{(2,3)}& (1-p)^2\,C_5 + p(1-p)\,(C_2 + C_4) + p^2(1-\tilde{p})\,C_6 + p^2\tilde{p}\,C_3.
\end{split}
\end{equation}
%
\begin{equation}
\begin{split}
C_6 \xrightarrow{(1,2)}& C_6\\
    \xrightarrow{(4,1)}& (1-p)\,C_6 + p\,C_1\\
    \xrightarrow{(3,4)}& (1-p)\,C_6 + p\,C_1\\
    \xrightarrow{(2,3)}& (1-p)^2\,C_6 + p(1-p)\,(C_1 + C_3) + p^2\,C_5.
\end{split}
\end{equation}

Матрицата на преход има вида
{\small
\begin{equation}
\begin{split}
&T_{1} = \\
&\begin{pmatrix}
 1-p & 0 & 0 & 0 & p & 0\cr
 0 & 1-p & p\,\tilde{p} & 0 & 0 & p\,\left( 1-\tilde{p}\right) \cr
 0 & 0 & 1-p & p\,\tilde{p} & p\,\left( 1-\tilde{p}\right)  & 0\cr
 \left( 1-p\right) \,p\,\tilde{p} & 0 & {p}^{2}\,\left( 1-\tilde{p}\right)  & 1-p & {p}^{2}\,\tilde{p} & \left( 1-p\right) \,p\,\left( 1-\tilde{p}\right) \cr
 0 & \left( 1-p\right) \,p & {p}^{2}\,\tilde{p} & \left( 1-p\right) \,p & {\left( 1-p\right) }^{2} & {p}^{2}\,\left( 1-\tilde{p}\right) \cr
 \left( 1-p\right) \,p & 0 & \left( 1-p\right) \,p & 0 & {p}^{2} & {\left( 1-p\right) }^{2}
\end{pmatrix}.
\end{split}
\end{equation}
}

Вероятностите $P_i$ за поява на състоянието $C_i$ можем за изразим в удобен вид чрез $P_5$
както следва
\begin{equation}
 P_2 = (1-p)\,P_5
\end{equation}
\begin{equation}
 P_3 = \frac{2(1-p) + p\tilde{p}}{2-p-\tilde{p}(1-\tilde{p})} P_5,
\end{equation}
\begin{equation}
 P_4 = \frac{(1-p)(2-p)+\tilde{p}(1-p+\tilde{p})}{2-p-\tilde{p}(1-\tilde{p})} P_5,
\end{equation}
\begin{equation}
 P_1 = \frac{(1-p)[2-2p+p^2+2\tilde{p}(\tilde{p}-p)]}{2-p-\tilde{p}(1-\tilde{p})} P_5,
\end{equation}
\begin{equation}
 P_6 = \frac{(1-\tilde{p})[2-2p+p^2+\tilde{p}(\tilde{p}-p)]}{2-p-\tilde{p}(1-\tilde{p})} P_5.
\end{equation}


\paragraph{Общата матрица на преход} $T$ приема вида
{\small
\begin{equation}
\begin{split}
&T = \\
&\begin{pmatrix}
 1-p  &\frac{3-p}{4} p\tilde{p} &0 &\frac{1-\tilde{p}}{4}{p}^{2} &p\frac{4+p\tilde{p}-3\tilde{p}-p}{4} &\frac{{p}^{2}\tilde{p}}{4}\cr
 \frac{1-\tilde{p}}{4}{p}^{2} &1-p  &\frac{3-p}{4} p\tilde{p} &0 &\frac{{p}^{2}\tilde{p}}{4} &p\frac{4+p\tilde{p}-3\tilde{p}-p}{4}\cr
 0 &\frac{1-\tilde{p}}{4}{p}^{2} &1-p  &\frac{3-p}{4} p\tilde{p} &p\frac{4+p\tilde{p}-3\tilde{p}-p}{4} &\frac{{p}^{2}\tilde{p}}{4}\cr
 \frac{3-p}{4} p\tilde{p} &0 &\frac{1-\tilde{p}}{4}{p}^{2} &1-p  &\frac{{p}^{2}\tilde{p}}{4} &p\frac{4+p\tilde{p}-3\tilde{p}-p}{4}\cr
 \frac{{p}^{2}\tilde{p}}{4} &(1-p) p &\frac{{p}^{2}\tilde{p}}{4} &(1-p) p &{\left( 1-p\right) }^{2} &\frac{2-\tilde{p}}{2}{p}^{2}\cr
 (1-p) p &\frac{{p}^{2}\tilde{p}}{4} &(1-p) p &\frac{{p}^{2}\tilde{p}}{4} &\frac{2-\tilde{p}}{2}{p}^{2} &{\left( 1-p\right) }^{2}
\end{pmatrix}.
\end{split}
\end{equation}
}

Вероятностите $P_i$ за поява на състоянието $C_i$ при произволна начална връзка можем за изразим в удобен вид
чрез параметър $r$ както следва:
\begin{equation}
 P_1=P_2=P_3=P_4 = r,
\end{equation}
\begin{equation}
 P_5 = P_6 = \frac{4 - p - 3\tilde{p} + 2p\tilde{p}}{4(1-p) + p\tilde{p}} r.
\end{equation}

От условието за нормиравка на вароятностите до единица $\sum_{i=1}^{6}P_i=4P_1+2P_5=1$,
определяме стойността на параметъра $r$ и получаваме следните крайни формули за вероятностите:
\begin{equation}
 P_1 = P_2 = P_3 = P_4 = \frac{1}{2}\;\frac{4(1-p) + p\tilde{p}}{12-9p-3\tilde{p}+4p\tilde{p}},
\end{equation}
\begin{equation}
 P_5 = P_6 = \frac{1}{2}\;\frac{4 - p - 3\tilde{p} + 2p\tilde{p}}{12-9p-3\tilde{p}+4p\tilde{p}}.
\end{equation}

За случая $p=\tilde{p}$ получаваме
\begin{equation}
 P_1 = \frac{1}{8}\;\frac{4(1-p) + p^2}{3(1-p) + p^2},
\end{equation}
\begin{equation}
 P_5 = \frac{1}{8}\;\frac{4(1-p) + 2p^2}{3(1-p) + p^2}.
\end{equation}

За случая $\tilde{p}=0$ получаваме
\begin{equation}
P_1 = \frac{1}{6}\;\frac{4-4p}{4-3p},
\end{equation}
\begin{equation}
P_5 = \frac{1}{6}\;\frac{4-p}{4-3p}.
\end{equation}
Тук за $p=1$ (т. е. твърдо движение на частиците) имаме $P_1=0$, $P_5=1/2$,
т. е. двучастичните състояния винаги ще се разпадат.

За случая $\tilde{p}=1$ получаваме
\begin{equation}
 P_1 = \frac{1}{2}\;\frac{4-3p}{9-5p},
\end{equation}
\begin{equation}
P_5 = \frac{1}{2}\;\frac{1+p}{9-5p}.
\end{equation}
Тук за $p=1/2$ получаваме $P_1=5/26$, $P_5=3/26$, т. е. при този
случай на ``залепване'' на частиците вероятностите за двучастичните
състояния (в интересната област $p\le1/2$) е по-голяма от вероятността
двете частици да не са съседни.
\end{document}
