%
% $Id: formal.tex,v 1.3 2003/04/02 21:35:06 e_gourgoulhon Exp $
%
\chapter{3+1 formalism with covariant conformal decomposition}

\section{Einstein equation}

We consider a 4-dimensional spacetime $\cal E$ endowed
with a Lorentzian metric $\w{g}$, which satisfies 
to the Einstein equation:
\be \label{e:einstein}
	{}^4\!R_{\alpha\beta} - {1\over 2} {}^4\!R g_{\alpha\beta} = 
	8\pi T_{\alpha\beta},
\ee
where
\begin{itemize}
\item ${}^4\!R_{\alpha\beta}$ is the Ricci tensor associated with $\w{g}$;
\item ${}^4\!R := {}^4\!R_\mu^{\ \, \mu}$ is the corresponding curvature scalar ;
\item $T_{\alpha\beta}$ is the matter energy-momentum tensor.
\end{itemize}
By convention, all Greek indices run in $\{0,1,2,3\}$.
In the following we denote by $\wg{\nabla}$ the covariant 
derivation associated with $\w{g}$. We will use letters from the
beginning of the alphabet ($\alpha$, $\beta$, ...) for free indices,
and letters starting from $\mu$ ($\mu$,$\nu$,$\rho$,...) as dumb indices
for contraction. In this way the tensorial degree (valence) of any 
equation is immediately apparent. 

\section{3+1 decomposition}

\subsection{Spacetime foliation}

The spacetime (or at least the part of it under study...)
is supposed to be foliated by a family of spacelike hypersurfaces $\Sigma_t$,
labeled by the time coordinate $t$.
We denote by $\w{n}$ the future directed unit normal to $\Sigma_t$.
By definition
$\w{n}$, considered as a 1-form, is parallel to the gradient of $t$:
\be \label{e:def_n}
	n_\alpha = - N \nabla_\alpha t . 
\ee
The proportionality factor $N$ is called the {\em lapse function}.
It ensures that $\w{n}$ satisfies to the normalization relation
\be
	n_\mu n^\mu = - 1  . 
\ee
The minus sign in (\ref{e:def_n}) is such that, when $N>0$, the 
vector $n^\alpha = g^{\alpha\mu} n_\mu$ (dual to the linear form 
(\ref{e:def_n}) via the metric) is oriented toward
increasing $t$'s.
We will call {\em Eulerian observer} the
observer whose 4-velocity is $\w{n}$.

\subsection{3-metric}

The metric $\wg{\gamma}$ induced by $\w{g}$ onto each hypersurface $\Sigma_t$
is given by the orthogonal projector onto $\Sigma_t$:
\be \label{e:def_3-metric}
	\gamma_{\alpha\beta} := g_{\alpha\beta} + n_\alpha n_\beta  .
\ee 
Since $\Sigma_t$ is assumed to be spacelike, $\wg{\gamma}$ is
a positive definite Riemannian metric. In the following, we
call it the {\em 3-metric}. 
Let us denote by $\wg{D}$ the covariant derivative
associated with $\wg{\gamma}$.
For any tensor $\w{T}$ lying in $\Sigma_t$ (i.e. such that
$\w{n}\cdot\w{T}=0$), $\wg{D}\w{T}$ can be expressed by
the full orthogonal projection of $\wg{\nabla}\w{T}$
onto $\Sigma_t$:
\be \label{e:nab_def}
D_\gamma T^{\alpha_1\ldots\alpha_p}_{\ \qquad\beta_1\ldots\beta_q}
		= \gamma_{\mu_1}^{\ \,\alpha_1} \, \cdots 
		 \gamma_{\mu_p}^{\ \,\alpha_p} \,
		  \gamma_{\beta_1}^{\ \,\nu_1} \, \cdots
		  \gamma_{\beta_q}^{\ \,\nu_q} \,
		  \gamma_\gamma^{\ \,\sigma} \, \nabla_\sigma
		  T^{\mu_1\ldots\mu_p}_{\ \qquad\nu_1\ldots\nu_q}  .		  	
\ee  

\subsection{Extrinsic curvature of $\Sigma_t$}

$\wg{\gamma}$ is also called the {\em first fundamental form}
of the hypersurface $\Sigma_t$. The {\em second fundamental form},
or {\em extrinsic curvature}, is given by the Lie derivative of
$\wg{\gamma}$ along the normal vector $\w{n}$:
\be \label{e:def_courb_extrins}
	K_{\alpha\beta} := - {1\over 2} \left( \pounds_{\wf{n}}\wg{\gamma}
		\right) _{\alpha\beta}  .
\ee
Expressing the Lie derivative in terms of $\wg{\nabla}$ 
and using (\ref{e:def_n}), we
get the identity
\be \label{e:K_grad_n}
	K_{\alpha\beta} = - \nabla_\alpha n_\beta - n_\alpha a_\beta  ,
\ee
where we have introduced the 4-acceleration of the Eulerian observer:
\be
	a_\alpha := n^\mu \nabla_\mu n_\alpha .
\ee 
Note that $\w{a}$ is orthogonal to $\w{n}$ (hence it lies in $\Sigma_t$),
and that (\ref{e:def_n}) gives rise to a relation between
$\w{a}$ and the 3-gradient of the lapse function:
\be \label{e:a_grad_N}
	a_\alpha = D_\alpha \ln N  .
\ee
\compar{Eq.~(\ref{e:K_grad_n}) agrees with Eq.~(27) of York 1979
\cite{York79} and Eq.~(\ref{e:a_grad_N}) agrees with Eq.~(28) of the
same reference.}

Eq.~(\ref{e:K_grad_n}) is very useful and we make a systematic use
of it to replace $\nabla_\alpha n_\beta$ in term of $K_{\alpha\beta}$.

We denote by $K$ the trace of $\w{K}$:
\be
	K := K_\mu^{\ \,\mu} =  - \nabla_\mu n^\mu  ,
\ee
where the second equality is a direct consequence of (\ref{e:K_grad_n}).

\subsection{Spatial coordinates and shift vector}

Let us introduce on each hypersurface $\Sigma_t$ a coordinate
system $x^i = (x^1,x^2,x^3)$ 
which varies smoothly between neighboring
hypersurfaces, so that $x^\alpha=(t,x^1,x^2,x^3)$ constitutes a 
well-behaved coordinate system of the whole spacetime\footnote{later on 
we will specify the coordinates $(x^i)$ to be of 
spherical type, with $x^1=r$, $x^2=\theta$ and $x^3=\phi$, but
at the present stage we keep $(x^i)$ fully general.}.  
We denote by 
\be
    \der{}{x^\alpha} = \left( \der{}{t}, \der{}{x^i} \right) 
    = \left( \der{}{t}, \der{}{x^1}, \der{}{x^2}, \der{}{x^3} \right)
\ee
the natural vector basis associated with this coordinate system. 
In the following Latin indices run in $\{1,2,3\}$.
 
The 3+1 decomposition of the basis vector $\partial /\partial t$ defines
the {\em shift vector} $\wg{\beta}$ of the spatial coordinates:
\be \label{e:dsdt_ortho}
	\left( \der{}{t} \right) ^\alpha 
		= N n^\alpha + \beta^\alpha
			\qquad \mbox{with} \qquad n_\mu \beta^\mu = 0 .
\ee
The fact that the part parallel to $n^\alpha$ is $N n^\alpha$
results from the duality relation 
$\nabla_\mu t \, \left( \der{}{t} \right) ^\mu = 1$ and
Eq.~(\ref{e:def_n}). 

As an immediate consequence of Eqs.~(\ref{e:def_n}) and
(\ref{e:dsdt_ortho}) the components of $\w{n}$ with respect
to the coordinates $(x^\alpha)$ are
\be \label{e:n_comp}
	n_\alpha = (-N,0,0,0) \qquad \mbox{and} \qquad 
	n^\alpha = \left( {1\over N}, -{\beta^1\over N},
		-{\beta^2\over N},-{\beta^3\over N} \right) .
\ee
Moreover, thanks to Eq.~(\ref{e:def_3-metric}), the 
components of the metric with respect to the coordinates 
$(x^\alpha)$ are
\be \label{e:g_cov}
	g_{\alpha\beta} = \left( \begin{array}{cc}
		g_{00} & g_{0j} \\
		g_{i0} & g_{ij}
		\end{array} \right) =
		\left( \begin{array}{cc}
		-N^2 + \beta_k \beta^k & \beta_j \\
		\beta_i & \gamma_{ij}
		\end{array} \right)
\ee
and
\be \label{e:g_con}
	g^{\alpha\beta} = \left( \begin{array}{cc}
		g^{00} & g^{0j} \\
		g^{i0} & g^{ij}
		\end{array} \right) =
		\left( \begin{array}{cc}
		-{1\over N^2} & {\beta^j\over N^2} \\
		{\beta^i\over N^2} & \gamma^{ij} 
		- {\beta^i\beta^j\over N^2}
		\end{array} \right) .
\ee
Hence the line element:
\be \label{e:line_g}
	g_{\mu\nu} \, dx^\mu\, dx^\nu
	= - N^2 dt^2 + \gamma_{ij} (dx^i + \beta^i dt)
		(dx^j + \beta^j dt) . 
\ee

\compar{Eq.~(\ref{e:n_comp}) agrees with Eqs.~(21.48) 
and (21.49) of MTW \cite{MisneTW73}, with the slight change
of notation $N^i_{\rm MTW} = \beta^i$; Eqs.~(\ref{e:g_cov}),
(\ref{e:g_con}) and (\ref{e:line_g}) agree with respectively
Eqs.~(21.42), (21.44) and (21.40) of MTW \cite{MisneTW73}.}

The Christoffel symbols of the covariant derivative $\wg{D}$
with respect to the coordinates $(x^i)$ are
\be \label{e:christo_gm}
	\Gamma^k_{\ \, ij} := {1\over 2} \gm^{kl} 
		\left( \partial_i\gm_{lj} +\partial_j \gm_{il}
			- \partial_l\gm_{ij} \right) ,
\ee
where we have used the abbreviation
\be
	\partial_i := \der{}{x^i} .
\ee



\subsection{3+1 decomposition of Einstein equation}

Let us first introduce the 3+1 decomposition of the energy-momentum
tensor:
\be
	T_{\alpha\beta} = S_{\alpha\beta} + n_\alpha J_\beta
		+ n_\beta J_\alpha + E n_\alpha n_\beta ,
\ee
with
\bea
	E &:=& T_{\mu\nu} n^\mu n^\nu ,\\
	J_\alpha & := & - \gm_\alpha^{\ \,\mu} T_{\mu\nu} n^\nu ,\\
	S_{\alpha\beta} & := & \gm_\alpha^{\ \,\mu} \gm_\beta^{\ \,\nu}
		T_{\mu\nu} .
\eea
Then, by contracting the Einstein equation (\ref{e:einstein})
by respectively $n^\mu n^\nu$, $\gm_\alpha^{\ \,\mu} n^\mu$ and
$\gm_\alpha^{\ \,\mu} \gm_\beta^{\ \,\nu}$, and making use of
Gauss and Codazzi relations (cf. e.g. Appendix A.1 of \cite{GourgB93}),
we obtain the following system of equations:
\be \label{e:ham_constr0}
	R + K^2 - K_{ij} K^{ij} = 16\pi E
\ee
\be \label{e:mom_constr0}
	D_j K_i^{\ \, j} - D_i K = 8\pi J_i
\ee
\be \label{e:evol_K0}
	\der{}{t}K_{ij} =  - D_i D_j N 
		+ N \left[ R_{ij} - 2 K_{ik} K^k_{\ j} + K K_{ij}
			+ 4\pi \left( (S-E)\gm_{ij} - 2 S_{ij} \right)
			\right] 
		+ \pounds_{\wg{\beta}} K_{ij} .
\ee 
\compar{Eqs.~(\ref{e:ham_constr0}), (\ref{e:mom_constr0})
and (\ref{e:evol_K0}) agree with respectively Eqs.~(23), (24) and
(39) of York 1979 \cite{York79}.}

Eq.~(\ref{e:ham_constr0}) is called
the {\em Hamiltonian constraint} and Eq.~(\ref{e:mom_constr0})
is called the {\em momentum constraint}. In the above equations 
\begin{itemize}
\item  $R_{ij}$ denotes the Ricci tensor
associated with the 3-metric $\wg{\gamma}$,
\item  $R=R_i^{\ \, i}$ the corresponding scalar curvature.
\end{itemize}

The Einstein equations (\ref{e:ham_constr0})-(\ref{e:evol_K0}) must
be supplemented by the kinematical relation (\ref{e:def_courb_extrins})
between $\w{K}$ and $\wg{\gm}$:
\be \label{e:kin0}
	\der{}{t}\gm_{ij} = - 2N K_{ij} + \pounds_{\wg{\beta}} \gm_{ij} ,
\ee
with
\be
	\pounds_{\wg{\beta}} \gm_{ij} = D_i \beta_j
		+ D_j \beta_i .	
\ee


\section{Conformal decomposition}

\subsection{Conformal 3-geometry}

Our aim is to separate the 3-metric $\wg{\gm}$ into a constrained
part [i.e. a part fixed by the Hamiltonian constraint (\ref{e:ham_constr0})],
and a unconstrained part which carries the dynamical degrees of freedom
of the gravitational field. York \cite{York72} has shown in 1972 
that this last role is played by the conformal ``metric''
\be \label{e:def_hatg}
	\hat \gm_{ij} := \gm^{-1/3} \, \gm_{ij} ,
\ee 
where
\be \label{e:def_detg}
	\gm := \det\gm_{ij}.
\ee
The quantity defined by (\ref{e:def_hatg}) is a tensor density of
weight $-2/3$, which has unit determinant and which is invariant in
any conformal transformation of $\gm_{ij}$. It can be seen as
representing the equivalence class of conformally related metrics
to which belongs the 3-metric $\wg{\gm}$. 
The constrained part of $\gm_{ij}$ is then
the determinant (\ref{e:def_detg}).

\subsection{Introduction of a flat metric}

One would like to introduce a connection $\tna_i$ which is associated
with the conformal geometry in the sense that 
\begin{itemize}
\item[(i)] $\tna_k \hat \gm_{ij} = 0$ ;
\item[(ii)] $\tna_i$ has a vanishing Riemann curvature if $\gm_{ij}$
is conformally flat.
\end{itemize}
It is easy to realize that property (i) is already 
satisfied by the connection $D_i$ associated with $\gm_{ij}$. This
follows from the basic formula for the variation of the determinant:
\be \label{e:variation_gm}
	\delta \ln\gm = \gm^{ij} \, \delta \gm_{ij} ,
\ee
which holds for any variation $\delta$ which obeys to the Leibnitz rule
(partial derivative, covariant derivative, Lie derivative, etc...).
Indeed, once applied to $\delta=D_i$, this formula results in
\be \label{e:nab_gm_zero}
	D_i \gm = 0 ,
\ee
from which property (i) follows with $\tna_i = D_i$. However $D_i$
does not satisfy to property (ii). So we seek for another connection,
which satisfy both (i) and (ii). Note in passing that the fact that
there might be two different connections, $D_i$ and $\tna_i$, satisfying
(i) reflects the fact that $\hat\gm_{ij}$ is not a true metric,
but a tensor density. In other words the unicity theorem of the 
associated connection to a given metric cannot be applied to $\hat\gm_{ij}$.

Finding the connection $\tna_i$
requires the introduction of an extra structure 
on $\Sigma_t$. Some authors introduce Cartesian coordinates
\cite{ShibaN95,BaumgS98}, but since we aim at using spherical coordinates,
we follow another path: we keep general coordinates and introduce
a flat metric $\w{f}$ on each hypersurface $\Sigma_t$ such that
\be \label{e:f_notime}
	\der{}{t} f_{ij} = 0 . 
\ee
{\sl ?? Condition topologique sur $\Sigma_t$ pour pouvoir introduire
une m\'etrique plate ??}

The inverse metric is denoted by $f^{ij}$:
\be
	f^{ik} f_{kj} = \delta^i_{\ \, j} . 
\ee
In particular note that, except for the very special case
$\gm_{ij}=f_{ij}$, one has
\be
	f^{ij} \not= \gm^{ik} \gm^{jl}\, f_{kl} .
\ee

We denote by $\wg{\cD}$ the covariant derivative associated
with $\w{f}$:
\be \label{e:Df_zero}
	\cD_k f_{ij} = 0 , 
\ee
and define
\be \label{e:def_upcD}
	\cD^i = f^{ij} \cD_i . 
\ee
The Christoffel symbols of the connection $\wg{\cD}$
with respect to the coordinates $(x^i)$ are denoted by 
$\bar\Gamma^k_{\ \, ij}$; they are given by the
standard expression:
\be \label{e:christo_f}
	\bar\Gamma^k_{\ \, ij} = {1\over 2} f^{kl} 
		\left( \partial_i f_{lj} +\partial_j f_{il}
			- \partial_l f_{ij} \right) .
\ee

Let us denote by $f$ the determinant of the matrix of the
covariant components of $\w{f}$:
\be \label{e:def_detf}
	f := \det f_{ij}.
\ee
Similarly to (\ref{e:variation_gm}), one has the following 
formula for any variation $\delta$ which obeys to the Leibnitz
rule:
\be \label{e:variation_f}
	\delta \ln f = f^{ij} \, \delta f_{ij} ,
\ee

\subsection{Conformal metric}

Instead of the tensor density (\ref{e:def_hatg}), we define
the {\em conformal metric} as being the tensor field
\be \label{e:def_metconf}
	\tgm_{ij} := \left( {\gm\over f} \right) ^{-1/3} \, \gm_{ij},
\ee
where $\gm$ is the determinant (\ref{e:def_detg}) and $f$
the determinant (\ref{e:def_detf}).
The fact that $\tgm_{ij}$ is a tensor follows from the fact
that the quotient $\gm/f$ is a scalar field on $\Sigma_t$. Indeed
a change of coordinates $(x^i) \mapsto (x^{i'})$ induces the following
changes in the determinants:
\bea
	\gm'  & = & (\det J)^2 \gm \label{e:change_g} \\
	f' & = & (\det J)^2 f ,\label{e:change_f}
\eea
where $J$ denotes the Jacobian matrix
\be
	J^i_{\ \, i'} := \der{x^i}{x^{i'}} .
\ee
From Eqs.~(\ref{e:change_g})-(\ref{e:change_f}) it is obvious
that $\gm'/f' = \gm /f$, which shows that $\gm/f$ is a scalar field. 
In the following we introduced the related scalar field:
\be
	\Psi := \left( {\gm\over f} \right)^{1/ 12} ,
\ee
so that the relation (\ref{e:def_metconf}) becomes
\be \label{e:def_tgm2}
	\tgm_{ij} = \Psi^{-4}\, \gm_{ij}
	\qquad \mbox{or} \qquad \gm_{ij} = \Psi^4 \, \tgm_{ij} .
\ee
$\Psi$ being always strictly positive (for $\gm$ and $f$ are 
strictly positive), $\tgm_{ij}$ is a Riemannian metric on $\Sigma_t$.
In fact it is the member of the conformal equivalence class of $\gm_{ij}$
which has the same determinant as the flat metric $f_{ij}$:
\be \label{e:dettgm_f}
	\det \tgm_{ij} = f .
\ee
In this respect, our approach agrees with the point of view of York in the
work \cite{York99}, who prefers to introduce a specific member of
the conformal equivalence class of $\gm_{ij}$ instead of manipulating
tensor densities such as (\ref{e:def_hatg}). In our case, we 
use the extra-structure $\w{f}$ to pick out
the representative member of the conformal equivalence class by the
requirement (\ref{e:dettgm_f}).

We define the {\em inverse conformal metric} $\tgm^{ij}$ by
the requirement
\be
	\tgm_{ik} \, \tgm^{kj} = \delta_i^{\ \, j} ,
\ee
which is equivalent to
\be
	\tgm^{ij}  = \Psi^4 \, \gm^{ij} \qquad \mbox{or} \qquad 
	\gm^{ij}  = \Psi^{-4} \, \tgm^{ij}.
\ee
Note also that although we are using the same notation $\tgm$
for both $\tgm_{ij}$ and $\tgm^{ij}$, one has
\be
	\tgm^{ij} \not= \gm^{ik} \gm^{jl}\, \tgm_{kl} ,
\ee
except in the special case $\Psi=1$.

The general formula for the variation of the determinant applied
to the matrix $\tgm_{ij}$ and combined with (\ref{e:dettgm_f})
gives
\be \label{e:variation_tgm}
	\delta \ln f = \tgm^{ij} \, \delta \tgm_{ij} ,
\ee
for any infinitesimal variation $\delta$ which obeys to Leibnitz rule. 
In the special case $\delta = \cD_k$, we deduce immediately that 
\be \label{e:variation_tgm2}
	\tgm^{ij} \cD_k \tgm_{ij} = 0 . 
\ee


\subsection{Conformal covariant derivation} 

Let us denote by $\wg{\tna}$ the connection associated with 
the metric $\tgm_{ij}$:
\be \label{e:tD_tgm_zero}
	\tna_k \tgm_{ij} = 0 . 
\ee
Its Christoffel symbols 
with respect to the coordinates $(x^i)$ are given by
the classical formula:
\be \label{e:christo_tgm}
	\tilde\Gamma^k_{\ \, ij} := {1\over 2} \tgm^{kl} 
		\left( \partial_i\tgm_{lj} +\partial_j \tgm_{il}
			- \partial_l\tgm_{ij} \right) .
\ee

The covariant derivatives $\wg{\tna}\w{T}$ and 
$\wg{\cD}\w{T}$ of any tensor field $\w{T}$ of type
$(q,p)$ on $\Sigma_t$ are related by the formula
\be \label{e:def_nablatilde}
	\tna_k T^{i_1\ldots i_p}_{\quad \quad j_1\ldots j_q}
	= \cD_k T^{i_1\ldots i_p}_{\quad \quad j_1\ldots j_q}
	+ \sum_{r=1}^p \Delta^{i_r}_{\ \, kl} \, 
		T^{i_1\ldots l \ldots i_p}_{\quad\quad \quad j_1\ldots j_q}
	- \sum_{r=1}^q \Delta^l_{\ \, k j_r} \, 
		T^{i_1\ldots i_p}_{\quad \quad j_1\ldots l \ldots j_q} ,
\ee  
where $\Delta$ denotes the following type (2,1) tensor field:
\be \label{e:def_Delta}
  \Delta^k_{\ \, ij} := {1\over 2} \tgm^{kl} 
		\left( \cD_i \tgm_{lj} + \cD_j \tgm_{il}
			- \cD_l\tgm_{ij} \right) .
\ee
$\Delta^k_{\ \, ij}$ can also be viewed as the difference between
the Christoffel symbols of $\tna_i$ and those of $\cD_i$:
\be
	\Delta^k_{\ \, ij} = \tilde\Gamma^k_{\ \, ij} - 
	\bar\Gamma^k_{\ \, ij} .	
\ee
(Recall that, whereas Christoffel symbols do not represent the 
components of any tensor field, the difference between two sets
of them does.)
 
As a direct consequence of (\ref{e:variation_tgm2}), we
have
\be
	\Delta^k_{\ \, ik} = 0 . 
\ee

The important property of $\wg{\tna}$ is that it coincides
with the flat covariant derivative $\wg{\cD}$ when the 3-metric is
conformally flat:
\bea
	\gm_{ij} = \Psi^4 f_{ij} & \Longrightarrow &
	\tgm_{ij} = f_{ij} \\
	& \Longrightarrow & \Delta^k_{\ \, ij} = 0 \\
	& \Longrightarrow & \wg{\tna} = \wg{\cD} . \label{e:conf_equalder}	
\eea

Applying the determinant variation formula (\ref{e:variation_tgm}) to 
$\delta = \tna_i$, we get immediately that the conformal covariant
derivative preserves the determinant of $f$ (as a tensor density
of weight 2):
\be
	\tna_i f = 0 .
\ee

Inserting (\ref{e:def_tgm2}) into (\ref{e:tD_tgm_zero}), we get
\be \label{e:td_gm}
	\tna_k \gm_{ij} = 4 \tna_k \ln\Psi \: \gm_{ij} .
\ee

Another property of $\wg{\tna}$ is that the divergence with respect
to it of any vector field $\w{V}$ coincides with the divergence with respect
to the flat covariant derivative $\wg{\cD}$:
\be
	\tna_k V^k = \cD_k V^k \ .
\ee
This follows from the expression of the divergence 
in terms of partial derivatives with
respect to the coordinates $(x^i)$:
\be
	\tna_k V^k = {1\over \sqrt{\tgm}}\, 
		\partial_k \left( \sqrt{\tgm} V^k \right)
	\qquad \mbox{and} \qquad
	\cD_k V^k = {1\over \sqrt{f}}\, 
		\partial_k \left( \sqrt{f} V^k \right)
\ee
and the fact that $\tgm = f$ [Eq.~(\ref{e:dettgm_f})].

In Eqs.~(\ref{e:evol_K0}) and (\ref{e:kin0}) appear Lie derivatives
along the shift vector. A useful formula in this respect is that
of the Lie derivative of the tensor density $f$:
\be
	\pounds_{\wg{\beta}} \ln f = 2 \tna_k \beta^k = 2 \cD_k \beta^k  .
\ee
This formula follows from Eq.~(\ref{e:variation_f}).

\subsection{Relation between the Ricci tensors of $\wg{D}$
and $\wg{\tna}$}

After some computation, we find the following relation between
the Ricci tensor $R_{ij}$ of the covariant derivative $\wg{D}$
(associated with the physical 3-metric $\wg{\gm}$) and the
Ricci tensor $\tilde R_{ij}$ of the covariant derivative $\wg{\tna}$
(associated with the conformal metric $\tgm_{ij}$):
\be \label{e:R_tR}
	R_{ij} = \tilde R_{ij} - 2 \tna_i \tna_j \ln \Psi 
	+ 4 \tna_i \ln\Psi \, \tna_j\ln\Psi
	- 2\left( \tna^k \tna_k\ln\Psi
	 + 2 \tna_k\ln\Psi \, \tna^k\ln\Psi \right)\, \tgm_{ij}  , 
\ee
where we have introduced the notation 
\be \label{e:def_uptD}
	\tna^i :=\tgm^{ij} \tna_j .
\ee
In particular note that the index of $\tna^i$ is not raised with 
$\gm^{ij}$. 

\compar{Eq.~(\ref{e:R_tR}) agrees with Eqs.~(2.15) and (2.16) of
Shibata \& Nakamura \cite{ShibaN95}, with $\phi = \ln\Psi + 1/12 \: \ln f$,
as well as with Eqs.~(19) and (20) of Baumgarte \& Shapiro \cite{BaumgS98}.
Note that these authors are using Cartesian coordinates only, so 
that $f=1$ for them.}

Let us recall that $\tilde R_{ij}$ vanishes identically for
a conformally flat 3-metric [cf. Eq.~(\ref{e:conf_equalder})]:
\be
	\gm_{ij} = \Psi^4 f_{ij} \; \Longrightarrow \; 
	\tilde R_{ij} = 0 .
\ee

Taking the trace of (\ref{e:R_tR}) results in
\be \label{e:trR_trtR}
	R = \Psi^{-4} \left( \tilde R 
	- 8 \tna_k \tna^k \ln\Psi
	- 8 \tna_k \ln\Psi \, \tna^k \ln\Psi \right) ,
\ee
where we have introduced the scalar curvature of the metric
$\tgm_{ij}$:
\be \label{e:tildeR}
	\tilde R:=\tgm^{ij} \tilde R_{ij} .
\ee
An equivalent form of (\ref{e:trR_trtR}) is
\be
	R = \Psi^{-4} \tilde R - 8 \Psi^{-5} \tna_k \tna^k \Psi.
\ee

\compar{This equation agrees with Eq.~(54) of York \cite{York79}.}

\subsection{Conformal decomposition of the extrinsic curvature}

We represent the traceless part of the extrinsic curvature
by the tensor field
\be
	A^{ij} := \Psi^{4} \left( K^{ij} - {1\over 3} K \gm^{ij}
		\right).
\ee
Let us define the following tensor field of type $(2,0)$ 
\be
	\taa_{ij} := \tgm_{ik} \tgm_{jl} A^{kl},
\ee
which can be seen as $A^{ij}$ with the indices lowered by 
$\tgm_{ij}$, instead of $\gm_{ij}$. To take care of this 
distinction, we have put a tilde atop $\taa_{ij}$.
$\taa_{ij}$ is related to $K_{ij}$ by the formula:
\be
	\taa_{ij} = \Psi^{-4} \left( K_{ij} - {1\over 3} K \gm_{ij}
		\right), 
\ee
Both $A^{ij}$ and $\taa_{ij}$ are traceless, in the sense that
\be
	\gm_{ij} A^{ij} = \tgm_{ij} A^{ij} = 0 \qquad \mbox{and} \qquad
	\gm^{ij} \taa_{ij} = \tgm^{ij} \taa_{ij} = 0 .
\ee

\subsection{Conformal decomposition of Einstein equations}

Thanks to (\ref{e:trR_trtR}), the Hamiltonian constraint 
(\ref{e:ham_constr0}) can be re-written
\be \label{e:ham_constr1}
   \tna_k\tna^k \ln \Psi + \tna_k\ln\Psi \tna^k\ln\Psi
   = {1\over 8} \tilde R - \Psi^4 \left( 2\pi E 
   	+ {1\over 8} \taa_{kl}A^{kl} 
	- {K^2\over 12} \right) .
\ee
An equivalent form of this equation is
\be 
   \tna_k\tna^k \Psi 
   = {1\over 8} \tilde R \, \Psi - \Psi^5 \left( 2\pi E 
   	+ {1\over 8} \taa_{kl}A^{kl} 
	- {K^2\over 12} \right) .
\ee

\compar{This equations agrees with Eq.~(70) of York \cite{York79}.}

The momentum constraint (\ref{e:mom_constr0}) becomes
\be \label{e:mom_constr1}
	\tna_j A^{ij} + 6 A^{ij} \tna_j\ln\Psi
		-{2\over 3} \tna^i K = 8\pi\Psi^4 J^i .
\ee

\compar{This equations agrees with Eq.~(44) of Alcubierre et al.
\cite{AlcubABSS00} in the special case of Cartesian coordinates;
these authors are using the quantity $\phi = \ln\Psi + 1/12\: \ln f$,
with $f=1$ in Cartesian coordinates.}

The trace of the dynamical equation (\ref{e:evol_K0}) [combined
with the Hamiltonian constraint (\ref{e:ham_constr0})] gives rise to
an evolution equation for the trace $K$ of the extrinsic curvature:
\be \label{e:evol_K1tr}
	\der{K}{t} =  \beta^k \tna_k K  - \Psi^{-4} \left(
	\tna_k\tna^k N + 2 \tna_k\ln\Psi \, \tna^k N \right)
	+ N\left[ 4\pi (E+S) + \taa_{kl} A^{kl}
	+ {K^2\over 3} \right] ,  
\ee
whereas the traceless part of (\ref{e:evol_K0}) becomes
\bea
  \der{A^{ij}}{t}  & = &  \pounds_{\wg{\beta}} A^{ij} +
  {2\over 3} \tna_k\beta^k \, A^{ij} - \Psi^{-6}
   \left( \tna^i\tna^j Q - {1\over 3} \tna_k \tna^k Q \: \tgm^{ij} 
   \right) \nonumber \\
  & & + \Psi^{-4} \Bigg\{ N \left(
  \tgm^{ik} \tgm^{jl} \tilde R_{kl} + 8 \tna^i \ln\Psi\, \tna^j \ln\Psi
  \right) + 4 \left( \tna^i\ln\Psi \, \tna^j N + \tna^j\ln\Psi \,
  \tna^i N \right) \nonumber \\
  & & - {1\over 3} \left[  N \left( \tilde R + 8 \tna_k\ln\Psi \tna^k\ln\Psi
  \right) + 8 \tna_k\ln\Psi \tna^k N \right] \, \tgm^{ij} \Bigg\} \nonumber \\
  & & + N \left[ K A^{ij} + 2 \tgm_{kl} A^{ik} A^{jl} 
   - 8\pi \left( \Psi^4 S^{ij} - {1\over 3} S \tgm^{ij} \right) 
  \right] , \label{e:evol_K1}
\eea
with 
\be
  \pounds_{\wg{\beta}} A^{ij} = \beta^k \tna_k A^{ij}
  	- A^{kj} \tna_k \beta^i - A^{ik} \tna_k \beta^j , 
\ee
and where we have introduced the scalar field
\be
	Q := \Psi^2 N .
\ee

Eqs.~(\ref{e:ham_constr1}) and (\ref{e:evol_K1tr}) can be combined into an
elliptic equation for $Q$:
\bea
   \tna_k \tna^k Q &=& \Psi^6 \left[ N \left( 4\pi S
   + {3\over 4} \taa_{kl}A^{kl} +{K^2\over 2} \right)
   - \der{K}{t} + \beta^k \tna_k K \right] \nonumber \\
   & & + \Psi^2 \left[  N \left( {3\over 2}\tilde R + 2
   	\tna_k \ln\Psi \, \tna^k \ln\Psi \right) 
	+ 2  \tna_k \ln\Psi \, \tna^k N \right] . \label{e:lapQ1}
\eea

The trace of the kinematical relation (\ref{e:kin0}) between
$\w{K}$ and $\wg{\gm}$ results in, thanks to (\ref{e:variation_gm}),
\be \label{e:kin1_tr}
   \der{\Psi}{t} = \beta^k\tna_k\Psi + {\Psi\over 6}
   \left( \tna_k\beta^k - N K \right) . 
\ee
The traceless part of (\ref{e:kin0}) results in
\be \label{e:kin1_st}
  \der{\tgm^{ij}}{t} = \pounds_{\wg{\beta}} \tgm^{ij} 
  + {2\over 3} \tna_k\beta^k \, \tgm^{ij} + 2N A^{ij} ,	
\ee
or equivalently
\be
  \der{\tgm^{ij}}{t} = 2N A^{ij} - \tna^i\beta^j - \tna^j\beta^i
  	+ {2\over 3} \tna_k\beta^k \, \tgm^{ij} .	
\ee



\section{Einstein equations in terms of the flat covariant derivative}

\subsection{Ricci tensor of $\wg{\tna}$ in terms of the
flat derivatives of $\wg{\tgm}$}

The Ricci tensor $\tilde R_{ij}$ of the covariant derivative $\wg{\tna}$ 
can be expressed in terms of the flat covariant derivatives of the 
conformal metric $\wg{\tgm}$ as
\bea
   \tilde R_{ij} & = & - {1\over 2} \tgm^{kl} \left( 
   \cD_k \cD_l \tgm_{ij} - \cD_k \cD_i \tgm_{lj}
   - \cD_k \cD_j \tgm_{il} \right) \nonumber \\
   & & + {1\over 2} \cD_k \tgm^{kl} \left( \cD_i \tgm_{lj} + \cD_j \tgm_{il}
   	- \cD_l \tgm_{ij} \right)
	- \Delta^k_{\ \, il} \Delta^l_{\ \, jk} . \label{e:ricci1}
\eea

\compar{This equation agrees with Eq.~(2.17) of Shibata \& Nakamura
\cite{ShibaN95}, provided it is restricted to Cartesian coordinates,
for which $\cD_i \rightarrow \partial_i$ and 
$\Delta^k_{\ \, ij} \rightarrow \tilde\Gamma^k_{\ \, ij}$.}

After some manipulations, (\ref{e:ricci1}) can be written as
\bea
   \tilde R_{ij}  & = & - {1\over 2} \tgm^{kl} \left( 
   \cD_k \cD_l \tgm_{ij} + \tgm_{ik} \cD_j H^k  + \tgm_{jk} \cD_j H^k
   + H^k \cD_k \tgm_{ij} + \cD_i \tgm^{kl} \cD_k \tgm_{lj}
   + \cD_j \tgm^{kl} \cD_k \tgm_{il} \right) \nonumber \\
    & &- \Delta^k_{\ \, il} \Delta^l_{\ \, jk} , \label{e:ricci2}
\eea
where we have introduced the vector field
\be \label{e:def_H}
	H^i := \cD_j \tgm^{ij} .
\ee
From Eq.~(\ref{e:def_Delta}), one can deduce an alternative
expression of $H^i$:
\be \label{e:H_Delta}
	H^i = - \tgm^{kl} \Delta^i_{\ \, kl} . 
\ee

\compar{When restricted to Cartesian coordinates ($\cD_i = \partial_i$,
$\Delta^i_{\ \, kl} = \tilde \Gamma^i_{\ \, kl}$), 
Eqs.~(\ref{e:def_H}) and (\ref{e:H_Delta}) agree 
with Eq.~(21) of Baumgarte \& Shapiro \cite{BaumgS98}, taking into
account their notation $\tilde\Gamma^i = - H^i$. 
Moreover, after some manipulations,
the expression (\ref{e:ricci2}) for the Ricci tensor can be shown to 
agree with Eq.~(22) of Baumgarte \& Shapiro \cite{BaumgS98}.}

Starting from (\ref{e:ricci2}), we obtain, after some computations,
an expression of the Ricci tensor in terms of the flat covariant derivatives
of $\tgm^{ij}$, instead of $\tgm_{ij}$:
\bea
  \tgm^{ik} \tgm^{jl} \tilde R_{kl} & = & {1\over 2} \Bigg(
  \tgm^{kl} \cD_k \cD_l \tgm^{ij} - \tgm^{ik} \cD_k H^j - \tgm^{jk} \cD_k H^i
  + H^k \cD_k\tgm^{ij} \nonumber \\
  &  & - \cD_l \tgm^{ik} \cD_k \tgm^{jl} 
  - \tgm_{kl} \tgm^{mn} \cD_m \tgm^{ik} \, \cD_n \tgm^{jl}
  + \tgm^{ik} \tgm_{ml} \cD_k \tgm^{mn} \, \cD_n \tgm^{jl} \nonumber \\
  & & + \tgm^{jl} \tgm_{kn} \cD_l \tgm^{mn} \, \cD_m \tgm^{ik} 
  +{1\over 2} \tgm^{ik} \tgm^{jl} \cD_k\tgm_{mn} \, \cD_l \tgm^{mn}
  \Bigg) \label{e:ricci3}.
\eea  

\compar{If we restrict ourselves to Cartesian coordinates,
the terms with second derivatives of $\tgm^{ij}$, i.e. the first
line of the above equation, agree with Eq.~(12) of Alcubierre et al.
\cite{AlcubBMS99}.}

The curvature scalar  $\tilde R$ defined from the Ricci
tensor $\tilde R_{ij}$ by (\ref{e:tildeR}) is basically the divergence
of $H^i$ plus some quadratic terms:
\be \label{e:tildeR_divH}
	\tilde R = - \cD_k H^k + {1\over 4} \tgm^{kl} \cD_k \tgm^{ij}
	\cD_l \tgm_{ij} - {1\over 2} \tgm^{kl} \cD_k \tgm^{ij} \cD_j \tgm_{il} . 
\ee
 

\subsection{Dirac gauge} \label{s:Dirac}

We define the {\em generalized Dirac gauge} by
\be
	H^i = 0 ,
\ee
or equivalently
\be
	\cD_j \left[ \left({\gm\over f} \right) ^{1/3} 
		\gm^{ij} \right] = 0 .
\ee
In Cartesian coordinates this choice reduces to the gauge introduced
by Dirac in 1959 \cite{Dirac59} as an attempt to fix the coordinates
in the Hamiltonian formulation of general relativity, preliminary
to its quantization (see \cite{Deser03} for a discussion):
\be
	\partial_j \left( \gm^{1/3} \gm^{ij} \right) = 0 . 
\ee

\compar{The above equation agrees with Eq.~(35) of Dirac \cite{Dirac59},
with Dirac's notation $\tilde e^{ij} = \tgm^{ij}$.}

Within the generalized Dirac gauge:
\begin{itemize}
\item the second order derivatives of
$\tgm^{ij}$ which appear in the expression (\ref{e:ricci3}) for the
Ricci tensor $\tilde R_{ij}$ reduce to the Laplacian-like term
$\tgm^{kl} \cD_k \cD_l \tgm^{ij}$;
\item the curvature scalar $\tilde R$ does not contain any
second order derivative of $\tgm^{ij}$.
\end{itemize}

A related coordinate choice is the ``Gamma freezing'' gauge
introduced by Alcubierre \& Br\"ugmann \cite{AlcubB01}.

\subsection{Einstein equations} \label{s:Einstein2}

The combination (\ref{e:lapQ1})
of the Hamiltonian constraint and the trace of
the spatial part of the dynamical Einstein equations 
can be re-written, taking into account (\ref{e:tildeR_divH}),
\bea
   \tgm^{kl}\cD_k \cD_l Q &=& - H^k \cD_k Q 
   + \Psi^6 \left[  N \left( 4\pi S
   + {3\over 4} \taa_{kl}A^{kl} +{K^2\over 2} \right) 
   - \der{K}{t} + \beta^k \cD_k K \right] \nonumber \\
   & & + \Psi^2 \Bigg[ N \left( - {3\over 2} \cD_k H^k 
   + {3\over 8} \tgm^{kl} \cD_k \tgm^{ij} \cD_l \tgm_{ij} 
   - {3\over 4} \tgm^{kl} \cD_k \tgm^{ij} \cD_j \tgm_{il}
   + 2 \tgm^{kl} \cD_k \ln\Psi \, \cD_l \ln\Psi \right) \nonumber \\
   & & \qquad + 2  \tgm^{kl}  \cD_k \ln\Psi \, \cD_l N \Bigg].	\label{e:lapQ2}
\eea
The momentum constraint (\ref{e:mom_constr1}) becomes
\be \label{e:mom_constr2}
	\cD_j A^{ij} + \Delta^i_{\ \, kl} A^{kl}
	+ 6 A^{ij} \cD_j\ln\Psi
		-{2\over 3} \tgm^{ij} \cD_j K = 8\pi\Psi^4 J^i .
\ee
The combination (\ref{e:evol_K1tr}) of the 
trace of the dynamical Einstein equations with the Hamiltonian
constraint equations writes
\bea 
	\der{K}{t} & = & \beta^k \cD_k K  - \Psi^{-4} \left(
	\tgm^{kl} \cD_k \cD_l N + H^k \cD_k N +
	 2 \tgm^{kl} \cD_k\ln\Psi \, \cD_l N \right) \nonumber \\
	& & + N\left[ 4\pi (E+S) + \taa_{kl} A^{kl}
	+ {K^2\over 3} \right] \label{e:evol_K2tr},  
\eea
whereas the traceless part becomes, thanks to (\ref{e:ricci3}):
\bea
	\der{A^{ij}}{t} & = & \Psi^{-4} \Bigg\{ {N\over 2}
	\Bigg[ \tgm^{kl} \cD_k \cD_l \tgm^{ij} - \tgm^{ik} \cD_k H^j
	-\tgm^{jk} \cD_k H^i + H^k \cD_k \tgm^{ij} \nonumber \\
   & & - \cD_l \tgm^{ik} \, \cD_k \tgm^{jl} 
   - \tgm_{kl} \tgm^{mn} \cD_m\tgm^{ik} \, \cD_n \tgm^{jl} 
   + \tgm^{ik} \tgm_{ml} \cD_k \tgm^{mn} \, \cD_n \tgm^{jl}
   + \tgm^{jl} \tgm_{kn} \cD_l \tgm^{mn} \, \cD_m \tgm^{ik} \nonumber \\
   &  & + \tgm^{ik} \tgm^{jl} \left( {1\over 2}  \cD_k \tgm_{mn} \cD_l \tgm^{mn}
   + 16  \cD_k\ln\Psi \, \cD_l \ln\Psi \right) \Bigg]
   - \Psi^{-2} \Bigg[ \tgm^{ik} \tgm^{jl} \cD_k \cD_l Q \nonumber \\
   &  & + {1\over 2} \left( \tgm^{il} \cD_l \tgm^{jk}
   	+ \tgm^{jl} \cD_l \tgm^{ik} - \tgm^{kl} \cD_l \tgm^{ij} \right) 
	\cD_k Q \Bigg]
	+ 4 \tgm^{ik} \tgm^{jl} \left( \cD_k \ln\Psi \cD_l N
		+ \cD_l \ln\Psi \cD_k N \right) \nonumber \\
   & & + {1\over 3} \Bigg[ \Psi^{-2} \left( \tgm^{kl} \cD_k \cD_l Q 
   + H^k \cD_k Q \right) - 8 \tgm^{kl} \cD_k\ln\Psi \cD_l N
   \nonumber \\
   & & + N \left( \cD_k H^k - {1\over 4} \tgm^{kl} \cD_k \tgm^{mn} \, 
	\cD_l \tgm_{mn} + {1\over 2} \tgm^{kl} \cD_k \tgm^{mn} \, \cD_n \tgm_{ml}
	- 8 \tgm^{kl} \cD_k \ln\Psi \, \cD_l \ln\Psi \right)
	\Bigg] \tgm^{ij} \Bigg\} \nonumber \\
  & & + N \left[ K A^{ij} + 2\tgm_{kl} A^{ik} A^{jl}
  	- 8 \pi \left( \Psi^4 S^{ij} - {1\over 3} S \tgm^{ij} 
	\right) \right] \nonumber \\
  & & + \beta^k \cD_k A^{ij} - A^{kj} \cD_k \beta^i - A^{ik} \cD_k \beta^j
		+ {2\over 3} \cD_k \beta^k \: A^{ij} \label{e:evol_K2} .
\eea 
These equations must be supplemented by the 
kinematical relations (\ref{e:kin1_tr}) 
\be 
   \der{\Psi}{t} = \beta^k\cD_k\Psi + {\Psi\over 6}
   \left( \cD_k\beta^k - N K \right) . \label{e:kin2_tr}
\ee
and (\ref{e:kin1_st}): 
\be
  \der{\tgm^{ij}}{t} = 2N A^{ij} + \beta^k\cD_k\tgm^{ij}
  	- \tgm^{ik} \cD_k \beta^j - \tgm^{jk} \cD_k \beta^i
  	+ {2\over 3} \cD_k\beta^k \, \tgm^{ij} . \label{e:kin2_st}	
\ee




\section{Einstein equations in terms of the
potentials $h^{ij}$}

\subsection{Definition of $h^{ij}$} \label{s:def_h}

Let us introduce the deviation $\w{h}$ of the inverse conformal
metric $\tgm^{ij}$ from the inverse flat metric, according to the formula
\be \label{e:def_h}
	\tgm^{ij} =: f^{ij} + h^{ij} .
\ee
$\w{h}$ is a tensor field onto $\Sigma_t$ which satisfies the
following property:
\be
	\mbox{$\wg{\gm}$ conformally flat} \iff \w{h} = 0 .
\ee
Note that $\w{h}$ is a tensor of type $(0,2)$ (``twice contravariant
tensor'' $h^{ij}$) and, except where otherwise stated, we will manipulate it
as such, without introducing any bilinear form (``twice covariant tensor''
$h_{ij}$) dual to it. In particular, note that, except under special
circumstances (e.g. $\w{h}=0$), 
\bea
	\tgm_{ij} & \not = & f_{ij} + \gm_{ik} \gm_{jl} h^{kl} , \\
	\tgm_{ij} & \not = & f_{ij} + \tgm_{ik} \tgm_{jl} h^{kl} ,\\
	\tgm_{ij} & \not = & f_{ij} + f_{ik} f_{jl} h^{kl} .
\eea 
The flat covariant derivatives of $\w{h}$ coincide with those
of $\tgm^{ij}$:
\be
	\cD_k \tgm^{ij} = \cD_k h^{ij}.
\ee
In particular the vector field $\w{H}$ introduced in Eq.~(\ref{e:def_H})
and from which the Dirac gauge is defined, 
is the divergence of $\w{h}$:
\be
	H^i = \cD_j h^{ij} .
\ee
Thanks to the splitting (\ref{e:def_h}), we can express the 
differential operator
$\tgm^{kl}\cD_k\cD_l$ which appears in the Einstein equations listed
in \S~\ref{s:Einstein2} as 
\be
	\tgm^{kl}\cD_k\cD_l = \Delta + h^{kl} \cD_k\cD_l ,
\ee
where $\Delta$ is the Laplacian operator associated with the flat
metric:
\be \label{e:def_flat_lap}
	\Delta := f^{kl}\cD_k\cD_l = \cD_k \cD^k .
\ee 
Let us recall that $\cD^k$ is defined by Eq.~(\ref{e:def_upcD}).

\subsection{Einstein equations}

The combination (\ref{e:lapQ2})
of the Hamiltonian constraint and the trace of
the spatial part of the dynamical Einstein equations 
becomes
\bea
  \Delta  Q &=& -h^{kl} \cD_k \cD_l Q - H^k \cD_k Q 
   + \Psi^6 \left[  N \left( 4\pi S
   + {3\over 4} \taa_{kl}A^{kl} +{K^2\over 2} \right) 
   - \der{K}{t} + \beta^k \cD_k K \right] \nonumber \\
   & & + \Psi^2 \Bigg[ N \Big( - {3\over 2} \cD_k H^k 
   + {3\over 8} \tgm^{kl} \cD_k h^{ij} \cD_l \tgm_{ij} 
   - {3\over 4} \tgm^{kl} \cD_k h^{ij} \cD_j \tgm_{il}
   + 2 \cD_k \ln\Psi \, \cD^k \ln\Psi  \nonumber \\
   & & \qquad + 2 h^{kl} \cD_k \ln\Psi \cD_l\ln\Psi \Big)
   + 2 \cD_k \ln\Psi \, \cD^k N + 2 h^{kl}  \cD_k \ln\Psi \, \cD_l N \Bigg].	
   \label{e:lapQ3}
\eea
The momentum constraint (\ref{e:mom_constr2}) writes
\be 
	\cD_j A^{ij} + \Delta^i_{\ \, kl} A^{kl}
	+ 6 A^{ij} \cD_j\ln\Psi
		-{2\over 3} \tgm^{ij} \cD_j K = 8\pi\Psi^4 J^i ,
		\label{e:mom_constr3}
\ee
with the following expression for $\Delta^i_{\ \, kl}$,
alternative to (\ref{e:def_Delta}):
\be
    \Delta^k_{\ \, ij}	= -{1\over 2} \left( \cD^k \tgm_{ij}
    	+ h^{kl} \cD_l \tgm_{ij} + \tgm_{il} \cD_j h^{kl}
	+  \tgm_{lj} \cD_i h^{kl} \right) .
\ee
The combination (\ref{e:evol_K2tr}) of the 
trace of the dynamical Einstein equations with the Hamiltonian
constraint equations becomes
\bea 
	\der{K}{t} & = & \beta^k \cD_k K  - \Psi^{-4} \left(
	\Delta N + h^{kl} \cD_k \cD_l N + H^k \cD_k N +
	+ 2 \cD_k\ln\Psi \, \cD^k  N 
	+  2 h^{kl} \cD_k\ln\Psi \, \cD_l N \right) \nonumber \\
	& & + N\left[ 4\pi (E+S) + \taa_{kl} A^{kl}
	+ {K^2\over 3} \right] , \label{e:evol_K3tr} 
\eea
whereas the traceless part (\ref{e:evol_K2}) becomes:
\bea
	\der{A^{ij}}{t} & = & \Psi^{-4} \Bigg\{ {N\over 2}
	\Bigg[ \Delta h^{ij} + h^{kl} \cD_k \cD_l h^{ij} 
	- \tgm^{ik} \cD_k H^j
	-\tgm^{jk} \cD_k H^i + H^k \cD_k h^{ij} \nonumber \\
   & & - \cD_l h^{ik} \, \cD_k h^{jl} 
   - \tgm_{kl} \tgm^{mn} \cD_m h^{ik} \, \cD_n h^{jl} 
   + \tgm^{ik} \tgm_{ml} \cD_k h^{mn} \, \cD_n h^{jl}
   + \tgm^{jl} \tgm_{kn} \cD_l h^{mn} \, \cD_m h^{ik} \nonumber \\
   &  & + \tgm^{ik} \tgm^{jl} \left( {1\over 2}  \cD_k \tgm_{mn} \cD_l h^{mn}
   + 16  \cD_k\ln\Psi \, \cD_l \ln\Psi \right) \Bigg]
   - \Psi^{-2} \Bigg[ \tgm^{ik} \tgm^{jl} \cD_k \cD_l Q \nonumber \\
   &  & + {1\over 2} \left( \tgm^{il} \cD_l h^{jk}
   	+ \tgm^{jl} \cD_l h^{ik} - \tgm^{kl} \cD_l h^{ij} \right) 
	\cD_k Q \Bigg]
	+ 4 \tgm^{ik} \tgm^{jl} \left( \cD_k \ln\Psi \cD_l N
		+ \cD_l \ln\Psi \cD_k N \right) \nonumber \\
   & & + {1\over 3} \Bigg[ \Psi^{-2} \left( \Delta Q + h^{kl} \cD_k \cD_l Q 
   + H^k \cD_k Q \right) - 8 \tgm^{kl} \cD_k\ln\Psi \cD_l N
   \nonumber \\
   & & + N \left( \cD_k H^k - {1\over 4} \tgm^{kl} \cD_k h^{mn} \, 
	\cD_l \tgm_{mn} + {1\over 2} \tgm^{kl} \cD_k h^{mn} \, \cD_n \tgm_{ml}
	- 8 \tgm^{kl} \cD_k \ln\Psi \, \cD_l \ln\Psi \right)
	\Bigg] \tgm^{ij} \Bigg\} \nonumber \\
  & & + N \left[ K A^{ij} + 2\tgm_{kl} A^{ik} A^{jl}
  	- 8 \pi \left( \Psi^4 S^{ij} - {1\over 3} S \tgm^{ij} 
	\right) \right] \nonumber \\
  & & + \beta^k \cD_k A^{ij} - A^{kj} \cD_k \beta^i - A^{ik} \cD_k \beta^j
		+ {2\over 3} \cD_k \beta^k \: A^{ij} . \label{e:evol_K3}
\eea 
These equations must be supplemented by the 
kinematical relations (\ref{e:kin2_tr}) 
\be 
   \der{\Psi}{t} = \beta^k\cD_k\Psi + {\Psi\over 6}
   \left( \cD_k\beta^k - N K \right) . \label{e:kin3_tr}
\ee
and (\ref{e:kin2_st}): 
\be
  \der{h^{ij}}{t} = 2N A^{ij} - \cD^i \beta^k - \cD^j \beta^i 
  	+ \beta^k\cD_k h^{ij}
  	- h^{ik} \cD_k \beta^j - h^{jk} \cD_k \beta^i
  	+ {2\over 3} \cD_k\beta^k \, \tgm^{ij} . \label{e:kin3_st}	
\ee
To get this last relation, use has been made of (\ref{e:f_notime}).

\section{Equations in maximal slicing and Dirac gauge}

\subsection{Simplifications induced by maximal slicing and Dirac gauge}

The maximal slicing and Dirac gauge coordinate conditions are
defined by (cf. Sec.~\ref{s:Dirac}):
\be
	K  =  0 ,
\ee
\be
	H^i  = \cD_j h^{ij} =  0 .
\ee

In this case, the combination (\ref{e:lapQ3})
of the Hamiltonian constraint and the trace of
the spatial part of the dynamical Einstein equations 
simplifies somewhat
\bea
  \Delta  Q &=& 
   \Psi^6   N \left( 4\pi S
   + {3\over 4} \taa_{kl}A^{kl}  \right) -h^{kl} \cD_k \cD_l Q 
    \nonumber \\
   & & + \Psi^2 \Bigg[ N \left( {3\over 8} \tgm^{kl} \cD_k h^{ij} \cD_l \tgm_{ij} 
   - {3\over 4} \tgm^{kl} \cD_k h^{ij} \cD_j \tgm_{il}
   + 2 \cD_k \ln\Psi \, \cD^k \ln\Psi 
   + 2 h^{kl} \cD_k \ln\Psi \cD_l\ln\Psi \right)  \nonumber \\
   &  & \qquad + 2 \cD_k \ln\Psi \, \cD^k N + 2 h^{kl}  \cD_k \ln\Psi \, \cD_l N \Bigg].	
   \label{e:lapQ4}
\eea

The momentum constraint (\ref{e:mom_constr3}) becomes
\be 
	\cD_j A^{ij} + \Delta^i_{\ \, kl} A^{kl}
	+ 6 A^{ij} \cD_j\ln\Psi = 8\pi\Psi^4 J^i .
		\label{e:mom_constr4}
\ee
The combination (\ref{e:evol_K3tr}) of the 
trace of the dynamical Einstein equations with the Hamiltonian
constraint equations becomes an elliptic equation for the lapse
function:
\be 
	\Delta N =  \Psi^4 N\left[ 4\pi (E+S) + \taa_{kl} A^{kl}
	\right] 
	 - h^{kl} \cD_k \cD_l N 
	- 2 \cD_k\ln\Psi \, \cD^k  N 
	-  2 h^{kl} \cD_k\ln\Psi \, \cD_l N  . \label{e:evol_K4tr} 
\ee
whereas the traceless part (\ref{e:evol_K3}) becomes:
\bea
	\der{A^{ij}}{t} & = & \Psi^{-4} \Bigg\{ {N\over 2}
	\Bigg[ \Delta h^{ij} + h^{kl} \cD_k \cD_l h^{ij} 
	 - \cD_l h^{ik} \, \cD_k h^{jl} 
   - \tgm_{kl} \tgm^{mn} \cD_m h^{ik} \, \cD_n h^{jl} \nonumber \\
  & & + \tgm^{ik} \tgm_{ml} \cD_k h^{mn} \, \cD_n h^{jl}
   + \tgm^{jl} \tgm_{kn} \cD_l h^{mn} \, \cD_m h^{ik} \nonumber \\
  & &  + \tgm^{ik} \tgm^{jl} \left( {1\over 2}  \cD_k \tgm_{mn} \cD_l h^{mn}
   + 16  \cD_k\ln\Psi \, \cD_l \ln\Psi \right) \Bigg]
   - \Psi^{-2} \Bigg[ \tgm^{ik} \tgm^{jl} \cD_k \cD_l Q \nonumber \\
   &  & + {1\over 2} \left( \tgm^{il} \cD_l h^{jk}
   	+ \tgm^{jl} \cD_l h^{ik} - \tgm^{kl} \cD_l h^{ij} \right) 
	\cD_k Q \Bigg]
	+ 4 \tgm^{ik} \tgm^{jl} \left( \cD_k \ln\Psi \cD_l N
		+ \cD_l \ln\Psi \cD_k N \right) \nonumber \\
   & & + {1\over 3} \Bigg[ \Psi^{-2} \left( \Delta Q + h^{kl} \cD_k \cD_l Q 
   + H^k \cD_k Q \right) - 8 \tgm^{kl} \cD_k\ln\Psi \cD_l N
   \nonumber \\
   & & + N \left( \cD_k H^k - {1\over 4} \tgm^{kl} \cD_k h^{mn} \, 
	\cD_l \tgm_{mn} + {1\over 2} \tgm^{kl} \cD_k h^{mn} \, \cD_n \tgm_{ml}
	- 8 \tgm^{kl} \cD_k \ln\Psi \, \cD_l \ln\Psi \right)
	\Bigg] \tgm^{ij} \Bigg\} \nonumber \\
  & & + N \left[ 2\tgm_{kl} A^{ik} A^{jl}
  	- 8 \pi \left( \Psi^4 S^{ij} - {1\over 3} S \tgm^{ij} 
	\right) \right] \nonumber \\
  & & + \beta^k \cD_k A^{ij} - A^{kj} \cD_k \beta^i - A^{ik} \cD_k \beta^j
		+ {2\over 3} \cD_k \beta^k \: A^{ij} . \label{e:evol_K4}
\eea 
The kinematical relation (\ref{e:kin3_tr}) becomes 
\be 
   \der{\Psi}{t} = \beta^k\cD_k\Psi + {\Psi\over 6}
   \cD_k\beta^k , \label{e:kin4_tr}
\ee
whereas (\ref{e:kin3_st}) remains unchanged: 
\be
  \der{h^{ij}}{t} = 2N A^{ij} - \cD^i \beta^j - \cD^j \beta^i 
  	+ \beta^k\cD_k h^{ij}
  	- h^{ik} \cD_k \beta^j - h^{jk} \cD_k \beta^i
  	+ {2\over 3} \cD_k\beta^k \, \tgm^{ij} . \label{e:kin4_st}	
\ee

\subsection{Expression of Dirac gauge on the shift vector}

Taking the (flat) divergence of (\ref{e:kin4_st}) and using the
fact that $\partial/\partial t$ commutes with $\cD_i$, thanks to 
(\ref{e:f_notime}), the Dirac gauge leads to
\be
   \cD_j A^{ij} = - {A^{ij}\over N} \cD_j N 
   + {1\over 2N} \left[ \Delta \beta^i + {1\over 3} \cD^i \left( \cD_j\beta^j
   \right) + h^{kl} \cD_k \cD_l \beta^i + {1\over 3} h^{ik} \cD_k
   \left( \cD_l \beta^l \right) \right] .
\ee
Inserting this relation into the momentum constraint equation
(\ref{e:mom_constr4})
leads to an elliptic equation for $\wg{\beta}$:
\bea
  \Delta\beta^i + {1\over 3} \cD^i\left(\cD_j\beta^j\right) & = &
   2 A^{ij} \cD_j N + 16\pi N \Psi^4 J^i - 12 N A^{ij} \cD_j\ln\Psi
  - 2 \Delta^i_{\ \, kl} N A^{kl} \nonumber \\
  & & - h^{kl} \cD_k \cD_l \beta^i - {1\over 3} h^{ik} \cD_k \cD_l \beta^l .
\eea

