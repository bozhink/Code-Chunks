%
% $Id: harmonics.tex,v 1.3 2003/06/16 11:57:20 j_novak Exp $
%
\chapter{Spherical tensor harmonics}
The evolution equations for the gravitational field can be written as
a tensor wave equation, when inserting Eq.~(\ref{e:kin3_st}) into the
evolution for the extrinsic curvature (\ref{e:evol_K3}). As for the
scalar wave equation, it may be useful to decompose the tensorial
field onto a basis of {\em tensorial spherical harmonics}. An extensive
review of these harmonics has been made by Thorne \cite{Thorn80} and we
shall use hereafter his definitions and notations.

\section{Pure orbital harmonics}

A first family of tensorial harmonics have been introduced by Mathews
\cite{Mathe62}, but the
complete set has been derived by Zerilli \cite{Zeril70}.
They are constructed from the canonical basis of rank-2 tensors,
multiplied with scalar spherical harmonics $Y_{lm}$ and combined so that they
transform under irreducible representations of $SO(3)$. The general
covariant function on the 2-sphere can be expanded in terms of these pure
orbital harmonics $\w{T}^{\lambda l', lm}$, which form an
orthonormal set and are divided into three categories:
\begin{itemize}
\item $\lambda=0$ gives the trace of the tensor, and $l'=l$ (one element);
\item $\lambda = 1$ gives the antisymmetric part of the tensor, and
$l'=l, l\pm1$ (3 elements);
\item $\lambda=2$ gives the symmetric trace-free part of the tensor,
and $l'=l,\  l\pm 1,\ l\pm 2 $ (5 elements).
\end{itemize}
They are eigenfunctions of the angular Laplacian 
$$
\Delta_{\th\ph} := \dder{}{\th} +{1\over\tan\th} \der{}{\th}
	+ {1\over\sin^2\th}\dder{}{\ph}
$$
so that 
\be
\Delta_{\th\ph} \w{T}^{\lambda l', lm} = - l'(l'+1) \w{T}^{\lambda l',
lm} . 
\ee
The general tensor solution of the Laplace's equation
\be
\Delta \w{U} = 0
\ee
is 
\be
\w{U}(r, \th, \ph) = \sum_{\lambda, l', l, m} \left( a^{\lambda
l', l m} \, r^{-(l'+1)} + b^{\lambda l', l m} \, r^{l'} \right)
\w{T}^{\lambda l', lm} .
\ee

\section{Pure spin harmonics}

Pure orbital harmonics are not well suited for describing radiation in
the radiation zone (where one supposes that the wave vector $\w{k}
\sim \w{e}_r$) because, under local rotation about the radial vector,
their tensor components do not transform as the components of the
polarization tensor. One can find linear combinations of the tensor
harmonics of a particular degree $l$ which are expressible in terms of
scalar spherical harmonics $Y_{lm}$ and their derivatives. This is
done using the following vector operators: $\w{e}_r, \wg{\nabla}$ and
$\w{L} = \w{e}_r \wedge \wg{\nabla}$ and obtaining another set of tensorial
harmonics $\w{T}^{J, lm}(\th,\ph )$
\begin{eqnarray}
\w{T}^{L_0, lm} &=& \w{e}_r \otimes \w{e}_r Y_{lm} \nonumber\\
\w{T}^{T_0, lm} &=& \frac{1}{\sqrt{2}} \left( \wg{\delta} -  \w{e}_r
\otimes \w{e}_r  \right) Y_{lm} \nonumber\\
\w{T}^{E_1, lm} &=& \sqrt{\frac{2}{l(l+1)}} \left( \w{e}_r \otimes
\wg{\nabla} Y_{lm} \right)^{\rm sym} \nonumber\\
\w{T}^{B_1, lm} &=& i\sqrt{\frac{2}{l(l+1)}} \left( \w{e}_r \otimes
\w{L} Y_{lm} \right)^{\rm sym} \nonumber\\
\w{T}^{E_2, lm} &=& \frac{1}{\sqrt{2l(l+1)(l-1)(l+2)}} \left(
\wg{\nabla} \otimes \wg{\nabla} Y_{lm} + 3 \w{e}_r \otimes \wg{\nabla}
Y_{lm} + \w{L} \otimes \w{L} Y_{lm} \right)^{\rm sym}  \nonumber\\
\w{T}^{B_2, lm} &=& \frac{i}{\sqrt{2l(l+1)(l-1)(l+2)}} \left(
\w{L} \otimes \wg{\nabla} Y_{lm} + \w{e}_r \otimes \w{L}
Y_{lm} \right)^{\rm sym} \label{e:zeril1}
\end{eqnarray}
where the superscript ``sym'' indicates the symmetric part of a
tensor. 

Regge and Wheeler \cite{ReggeW57} have introduced a very similar
family of spherical 
harmonics, with the difference that they used a combination of
$\w{T}^{E_2, lm}$ and $\w{T}^{T_0, lm}$ instead of $\w{T}^{E_2, lm}$,
making their basis a non-orthonormal one. The set defined in
(\ref{e:zeril1}) is an orthonormal basis showing all components of a
tensor:
\begin{itemize}
\item $\w{T}^{L_0, lm}$ is spin 0, pure longitudinal;
\item $\w{T}^{T_0, lm}$ is spin 0, pure transverse and proportional to the
transverse projection tensor;
\item $\w{T}^{E_1, lm}$ and $\w{T}^{B_1, lm}$ are spin 1, mixed
longitudinal and transverse;
\item $\w{T}^{E_2, lm}$ and $\w{T}^{B_2, lm}$ are spin 2, transverse
and traceless (they carry the gravitational wave in the wave zone).
\end{itemize}
One can note that spin 1 elements ($\w{T}^{E_1, lm}$ and
$\w{T}^{B_1, lm}$) are only defined for $l \geq 1$, and spin 2 elements
($\w{T}^{E_2, lm}$ and $\w{T}^{B_2, lm}$) for $l \geq 2$. Finally, one
can write explicitly:
\begin{eqnarray*}
%
        {}[\w{T}^{L_0, lm}]_{ij} &=&
        \left( \begin{array}{ccc}
        1 & 0 & 0 \\
        0 & 0 & 0 \\
        0 & 0 & 0 \\
        \end{array} \right) Y_{lm} \\ 
%
        {}[\w{T}^{E_1, lm}]_{ij} &=&
        \frac{1}{\sqrt{2l(l+1)}}
        \left( \begin{array}{ccc}
        0 & \df{}{\th} & \frac{1}{\sin\th}\df{}{\ph} \\
        \df{}{\th}& 0 & 0 \\
        \frac{1}{\sin\th}\df{}{\ph} & 0 & 0 \\
        \end{array} \right) Y_{lm} \\ 
%
        {}[\w{T}^{B_1, lm}]_{ij} &=& 
        \frac{i}{\sqrt{2l(l+1)}}
        \left( \begin{array}{ccc}
        0 & \frac{1}{\sin\th} \df{}{\ph} & 
        -\df{}{\th} \\
        \frac{1}{\sin\th} \df{}{\ph} & 0 & 0 \\
        -\df{}{\th} & 0 & 0 \\
        \end{array} \right) Y_{lm} \\ 
%
        {}[\w{T}^{T_0, lm}]_{ij} &=&
        \frac{1}{\sqrt{2}}\left( \begin{array}{ccc}
        0 & 0 & 0 \\
        0 & 1 & 0 \\
        0 & 0 & 1 \\
        \end{array} \right) Y_{lm} \\ 
%
        {}[\w{T}^{E_2, lm}]_{ij} &=&
        \frac{1}{\sqrt{2l(l+1)(l-1)(l+2)}}
        \left( \begin{array}{ccc}
	0 & 0 & 0 \\
        0 & W_{lm} & X_{lm} \\
        0 & X_{lm} & -W_{lm} \\
        \end{array} \right) \;,\\ 
%
        {}[\w{T}^{B_2, lm}]_{ij} &=&
        \frac{i}{\sqrt{2l(l+1)(l-1)(l+2)}}
        \left( \begin{array}{ccc}
        0 & 0 & 0 \\
        0 & X_{lm} & -W_{lm} \\
        0 & -W_{lm} & -X_{lm} \\
        \end{array} \right)  
\end{eqnarray*}
with 
\bea
        X_{lm} &:=& \frac{2}{\sin \th}\left(\df{}{\th} -
        \cot\th\right)\df{}{\ph} Y_{lm} \\
        W_{lm} &:=& \left(\dff{}{\th} - \cot\th\df{}{\th}
        - \frac{1}{\sin^2\theta}\dff{}{\ph}\right) Y_{lm}
        =\left(l(l+1) + 2\dff{}{\th}\right)Y_{lm}.
\eea

These pure spin harmonics are not eigenfunctions of the angular
Laplacian. 

\section{Relations between both types of harmonics}

Following Zerilli \cite{Zeril70} and Thorne \cite{Thorn80}, the
relation between pure spin and pure orbital harmonics is: 
\bea
\w{T}^{L_0, lm} &=& \sqrt{\frac{(l+1)(l+2)}{(2l+1)(2l+3)}} \w{T}^{2\
l+2, lm} - \sqrt{\frac{2l(l+1)}{3(2l-1)(2l+3)}} \w{T}^{2\
l, lm} + \sqrt{\frac{l(l-1)}{(2l-1)(2l+1)}} \w{T}^{2\
l-2, lm} \nonumber\\
&&-\frac{1}{\sqrt{3}} \w{T}^{0\ l, lm},\nonumber\\
\w{T}^{T_0, lm} &=& - \sqrt{\frac{(l+1)(l+2)}{2(2l+1)(2l+3)}} \w{T}^{2\
l+2, lm} + \sqrt{\frac{l(l+1)}{3(2l-1)(2l+3)}} \w{T}^{2\
l, lm} - \sqrt{\frac{l(l-1)}{2(2l-1)(2l+1)}} \w{T}^{2\
l-2, lm} \nonumber\\
&&-\sqrt{\frac{2}{3}} \w{T}^{0\ l, lm}, \nonumber\\
\w{T}^{E_1, lm} &=& - \sqrt{\frac{2l(l+2)}{(2l+1)(2l+3)}} \w{T}^{2\
l+2, lm} - \sqrt{\frac{3}{(2l-1)(2l+3)}} \w{T}^{2\
l, lm} + \sqrt{\frac{2(l-1)(l+1)}{(2l-1)(2l+1)}} \w{T}^{2\
l-2, lm}, \nonumber\\
\w{T}^{B_1, lm} &=& i\sqrt{\frac{l+2}{2l+1}} \w{T}^{2\ l+1, lm} - i
\sqrt{\frac{l-1}{2l+1}} \w{T}^{2\ l-1, lm}, \nonumber\\ 
\w{T}^{E_2, lm} &=& \sqrt{\frac{l(l-1)}{2(2l+1)(2l+3)}} \w{T}^{2\
l+2, lm} + \sqrt{\frac{3(l-1)(l+2)}{(2l-1)(2l+3)}} \w{T}^{2\
l, lm} + \sqrt{\frac{(l+1)(l+2)}{2(2l-1)(2l+1)}} \w{T}^{2\ l-2, lm},
\nonumber\\ 
\w{T}^{B_2, lm} &=& -i\sqrt{\frac{l-1}{2l+1}} \w{T}^{2\ l+1, lm} - i
\sqrt{\frac{l+2}{2l+1}} \w{T}^{2\ l-1, lm}. \label{e:mz2rw}
\eea
This transformation is unitary and is ``easily'' inverted:
\bea
\w{T}^{0\ l, lm} &=& -\sqrt{3} \w{T}^{L_0, lm} - \sqrt{\frac{2}{3}}
\w{T}^{T_0, lm}, \nonumber \\
\w{T}^{2\ l-2, lm} &=& \sqrt{\frac{(l-1)l}{(2l-1)(2l+1)}} \w{T}^{L_0,
lm} - \sqrt{\frac{(l-1)l}{2(2l-1)(2l+1)}} \w{T}^{T_0, lm} +
\sqrt{\frac{2(l-1)(l+1)}{(2l-1)(2l+1)}} \w{T}^{E_1, lm} \nonumber\\
&&+ \sqrt{\frac{(l+1)(l+2)}{2(2l-1)(2l+1)}} \w{T}^{E_2, lm}, \nonumber \\
\w{T}^{2\ l-1, lm} &=& i\sqrt{\frac{l-1}{2l+1}} \w{T}^{B_1, lm} + i
\sqrt{\frac{l+2}{2l+1}} \w{T}^{B_2, lm}, \nonumber\\
\w{T}^{2\ l, lm} &=& - \sqrt{\frac{2l(l+1)}{3(2l-1)(2l+3)}}
\w{T}^{L_0, lm} + \sqrt{\frac{l(l+1)}{3(2l-1)(2l+3)}} \w{T}^{T_0, lm}
- \sqrt{\frac{3}{(2l-1)(2l+3)}} \w{T}^{E_1, lm} \nonumber\\
&&+ \sqrt{\frac{3(l-1)(l+2)}{(2l-1)(2l+3)}} \w{T}^{E_2, lm}, \nonumber\\
\w{T}^{2\ l+1, lm} &=& - i\sqrt{\frac{l+2}{2l+1}} \w{T}^{B_1, lm} + i
\sqrt{\frac{l-1}{2l+1}} \w{T}^{B_2, lm}, \nonumber\\
\w{T}^{2\ l+2, lm} &=& \sqrt{\frac{(l+1)(l+2)}{(2l+1)(2l+3)}}
  \w{T}^{L_0, lm} - \sqrt{\frac{(l+1)(l+2)}{2(2l+1)(2l+3)}}
  \w{T}^{T_0, lm} - \sqrt{\frac{2l(l+2)}{(2l+1)(2l+3)}} \w{T}^{E_1,
  lm} \nonumber\\
&&+ \sqrt{\frac{l(l-1)}{2(2l+1)(2l+3)}} \w{T}^{E_2, lm}. \label{e:rw2mz}
\eea

Having in mind the fact that the coefficients of these decompositions
depend on the quantic number $l$, one can see that in the general
3-dimensional case, all the components of a tensor describing a wave
and having the same parity are mixed (longitudinal / transverse). This
is due to the fact that the pure spin harmonics are not eigenfunctions
of the angular Laplacian. There can be a decoupling of all modes, when
applying this analysis to the description of perturbations on a
spherically symmetric space-time (e.g. Schwarzschild, like the work by
Regge and Wheeler \cite{ReggeW57}). But, in a more general case, and
when using numerical methods to tackle the problem, this
decompositions may not be the best suited. Still, they give very good
insight on the asymptotic wave structure, where the wave vector
$\w{k}$ is collinear to $\w{e}_r$. 